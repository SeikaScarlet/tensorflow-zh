%!TEX program = xelatex
% Encoding: UTF8
% SEIKA 2015

%\documentclass[a4paper,11pt,twoside]{book}
\documentclass[a4paper,11pt,twoside]{ctexbook}

\usepackage{makeidx}
\makeindex

\usepackage{geometry}
\geometry{left=3.5cm, right=3cm, top=3cm, bottom=3cm}
%控制页眉页脚页码
\pagestyle{headings}
%罗马字符页码
%\pagenumbering{roman}

% \usepackage{ctex}
% \usepackage{xeCJK}

% \CJKsetecglue{} % 禁用汉字与其他内容之间空格(空隙)

% 支持西文字体
\usepackage{fourier}
\usepackage{courier}
% \usepackage{fontspec}

\newfontfamily\CodeFont{Consolas}
% \newfontfamily\CodeFont{Ubuntu Mono}
% \newfontfamily\CodeFont{Menlo}
% \newfontfamily\CodeFont{Lucida Console}
% \setmonofont{Lucida Console}

\usepackage{graphicx}
% 支持插入eps图形文件
% \usepackage{epsfig}

% 支持代码框插入
\usepackage{xcolor}
\definecolor{mygreen}{rgb}{0,0.6,0}
\definecolor{mygray}{rgb}{0.5,0.5,0.5}
\definecolor{mymauve}{rgb}{0.58,0,0.82}
\definecolor{codeback}{rgb}{0.8,0.87,0.95}
\definecolor{etc}{rgb}{0.24,0.07,0.18}
\definecolor{etc}{rgb}{0.54,0.37,0.48}

\usepackage{amsmath}

% 支持超链接
\usepackage[colorlinks]{hyperref}

\usepackage{listings}
\lstset{ %
  backgroundcolor=\color{codeback},    % choose the background color; you must add \usepackage{color} or \usepackage{xcolor}
  basicstyle=\linespread{0.95}\footnotesize\CodeFont,   % the size of the fonts that are used for the code
  breakatwhitespace=false,           % sets if automatic breaks should only happen at whitespace
  breaklines=true,                   % sets automatic line breaking
  captionpos=bl,                     % sets the caption-position to bottom
  commentstyle=\color{mygreen},      % comment style
  deletekeywords={...},              % if you want to delete keywords from the given language
  escapeinside={\%*}{*)},            % if you want to add LaTeX within your code
  extendedchars=true,                % lets you use non-ASCII characters; for 8-bits encodings only, does not work with UTF-8
  frame=single,                      % adds a frame around the code
  frameround=tttt,
  keepspaces=true,                   % keeps spaces in text, useful for keeping indentation of code (possibly needs columns=flexible)
  keywordstyle=\color{blue},         % keyword style
  language=Python,                   % the language of the code
  morekeywords={*,...},              % if you want to add more keywords to the set
  numbers=left,                      % where to put the line-numbers; possible values are (none, left, right)
  numbersep=4pt,                     % how far the line-numbers are from the code
  numberstyle=\tiny\CodeFont\color{mygray},   % the style that is used for the line-numbers
  rulecolor=\color{mygray},          % if not set, the frame-color may be changed on line-breaks within not-black text (e.g. comments (green here))
  showspaces=false,                  % show spaces everywhere adding particular underscores; it overrides 'showstringspaces'
  showstringspaces=true,             % underline spaces within strings only
  showtabs=true,                     % show tabs within strings adding particular underscores
  stepnumber=1,                      % the step between two line-numbers. If it's 1, each line will be numbered
  stringstyle=\color{orange},        % string literal style
  tabsize=2,                         % sets default tabsize to 2 spaces
  %title=myPython.py                 % show the filename of files included with \lstinputlisting; also try caption instead of title
  xleftmargin = 2em,
  xrightmargin = 2em,
  aboveskip = 0.5 em
}

% \setCJKmainfont[BoldFont={SimSun},ItalicFont={KaiTi}] %{SimSun}

%%%%%%%%%%%%
\title{TensorFlow 指南}
\author{}
\date{\today}

% \thanks{}

\begin{document}

\maketitle
\tableofcontents

%%%% 第一章
\newpage
\chapter{起步}
% \section{Introduction}
\include{get_started/c1s01_introduction}
%!TEX program = xelatex
% Encoding: UTF8
% SEIKA 2016 | seika@live.ca

% Chapter 1
% Section 1.2 OS_Setup


\section {Download and Setup   ||   下载与安装} \label{download_install}

Ⓔ \textcolor{etc}{You can install TensorFlow either from our provided binary packages or from the github source.}

Ⓒ 您可以使用我们提供的二进制包,或者源代码,安装 TensorFlow.

%
%%
\subsection {Requirements  |  安装需求}

Ⓔ \textcolor{etc}{The TensorFlow Python API currently supports Python 2.7 and Python 3.3+ from source.}

Ⓒ TensorFlow Python API 目前支持 Python 2.7 和 python 3.3以上版本.

Ⓔ \textcolor{etc}{The GPU version (Linux only) currently requires the Cuda Toolkit 7.0 and CUDNN 6.5 V2. Please see \hyperref[install_cuda]{Cuda installation}.}

Ⓒ 支持 GPU 运算的版本 (仅限Linux) 需要 Cuda Toolkit 7.0 和 CUDNN 6.5 V2. 具体请参考\hyperref[install_cuda]{Cuda安装}.

%
%%
\subsection {Overview  |  安装总述}

We support different ways to install TensorFlow:

TensorFlow 支持通过以下不同的方式安装:

\begin{itemize}
\item \hyperref[pip_install]{Pip Install}: \textcolor{etc}{Install TensorFlow on your machine, possibly upgrading previously installed Python packages. May impact existing Python programs on your machine.}\\
\item \hyperref[pip_install]{Pip 安装}: 在你的机器上安装TensorFlow,可能会同时更新之前安装的Python包,并且影响到你机器当前可运行的Python程序.\\
\item \hyperref[virtualenv_install]{Virtualenv Install}: \textcolor{etc}{Install TensorFlow in its own directory, not impacting any existing Python programs on your machine.}\\
\item \hyperref[virtualenv_install]{Virtualenv 安装}: 在一个独立的路径下安装TensorFlow,不会影响到你机器当前运行的Python程序.\\
\item \hyperref[docker_install]{Docker Install}: \textcolor{etc}{Run TensorFlow in a Docker container isolated from all other programs on your machine.}\\
\item \hyperref[docker_install]{Docker 安装}: 在一个独立的Docker容器中安装TensorFlow,并且不会影响到你机器上的任何其他程序.
\end{itemize}

Ⓔ \textcolor{etc}{If you are familiar with Pip, Virtualenv, or Docker, please feel free to adapt the instructions to your particular needs. The names of the pip and Docker images are listed in the corresponding installation sections.}

Ⓒ 如果你已经很熟悉Pip、Virtualenv、Docker这些工具的使用,请利用教程中提供的代码,根据你的需求安装TensorFlow.你会在下文的对应的安装教程中找到Pip或Docker安装所需的镜像.

Ⓔ \textcolor{etc}{If you encounter installation errors, see common problems for some solutions.}

Ⓒ 如果你遇到了安装错误,请参考章节\hyperref[comm_prob]{常见问题}寻找解决方案.

%
%%
\subsection {Pip Installatioin  |  Pip 安装} \label{pip_install}

Ⓔ \textcolor{etc}{\href{https://en.wikipedia.org/wiki/Pip_(package_manager)}{Pip} is a package management system used to install and manage software packages written in Python.}

Ⓒ \href{https://en.wikipedia.org/wiki/Pip_(package_manager)}{Pip} 是一个用于安装和管理Python软件包的管理系统.

Ⓔ \textcolor{etc}{The packages that will be installed or upgraded during the pip install are listed in the \href{https://github.com/tensorflow/tensorflow/blob/master/tensorflow/tools/pip_package/setup.py}{REQUIRED\_PACKAGES section of setup.py}}

Ⓒ 安装依赖包(\href{https://github.com/tensorflow/tensorflow/blob/master/tensorflow/tools/pip_package/setup.py}{REQUIRED\_PACKAGES section of setup.py}) 列出了pip安装时将会被安装或更新的库文件.

Ⓔ \textcolor{etc}{Install pip (or pip3 for python3) if it is not already installed:}

Ⓒ 如果pip尚未被安装,请使用以下代码先安装pip(如果你使用的是Python 3请安装 pip3 ):

\begin{lstlisting}[language = bash]
# Ubuntu/Linux 64-bit
$ sudo apt-get install python-pip python-dev
\end{lstlisting}

\begin{lstlisting}
# Mac OS X
$ sudo easy_install pip
\end{lstlisting}

Ⓔ \textcolor{etc}{Install TensorFlow:}

Ⓒ 安装 TensorFlow:

\begin{lstlisting}
# Ubuntu/Linux 64-bit, CPU only:
$ sudo pip install --upgrade https://storage.googleapis.com/tensorflow/linux/cpu/tensorflow-0.6.0-cp27-none-linux_x86_64.whl
\end{lstlisting}

\begin{lstlisting}
# Ubuntu/Linux 64-bit, GPU enabled:
$ sudo pip install --upgrade https://storage.googleapis.com/tensorflow/linux/gpu/tensorflow-0.6.0-cp27-none-linux_x86_64.whl
\end{lstlisting}

\begin{lstlisting}[language = bash]
# Mac OS X, CPU only:
$ sudo easy_install --upgrade six
$ sudo pip install --upgrade https://storage.googleapis.com/tensorflow/mac/tensorflow-0.6.0-py2-none-any.whl
\end{lstlisting}

Ⓔ \textcolor{etc}{For Python 3:}

Ⓒ 基于 Python 3 的 TensorFlow 安装:

\begin{lstlisting}
# Ubuntu/Linux 64-bit, CPU only:
$ sudo pip3 install --upgrade https://storage.googleapis.com/tensorflow/linux/cpu/tensorflow-0.6.0-cp34-none-linux_x86_64.whl
\end{lstlisting}

\begin{lstlisting}
# Ubuntu/Linux 64-bit, GPU enabled:
$ sudo pip3 install --upgrade https://storage.googleapis.com/tensorflow/linux/gpu/tensorflow-0.6.0-cp34-none-linux_x86_64.whl

\end{lstlisting}

\begin{lstlisting}[language = bash]
# Mac OS X, CPU only:
$ sudo easy_install --upgrade six
$ sudo pip3 install --upgrade https://storage.googleapis.com/tensorflow/mac/tensorflow-0.6.0-py3-none-any.whl
\end{lstlisting}

Ⓔ \textcolor{etc}{You can now test your \hyperref[test_install]{installation}.}

Ⓒ 至此你可以\hyperref[test_install]{测试安装}是否成功.


%
%%
\subsection {Virtualenv installation  |  基于 Virtualenv 安装} \label{virtualenv_install}

Ⓔ \textcolor{etc}{\href{http://docs.python-guide.org/en/latest/dev/virtualenvs/}{Virtualenv} is a tool to keep the dependencies required by different Python projects in separate places. The Virtualenv installation of TensorFlow will not override pre-existing version of the Python packages needed by TensorFlow.}

Ⓒ \href{http://docs.python-guide.org/en/latest/dev/virtualenvs/}{Virtualenv} 是一个管理在不同位置存放和调用 Python 项目所需依赖库的工具. TensorFlow 的 Virtualenv 安装不会覆盖先前已安装的 TensorFlow Python依赖包.

Ⓔ \textcolor{etc}{With \href{https://pypi.python.org/pypi/virtualenv}{Virtualenv} the installation is as follows:}

Ⓒ 基于\href{https://pypi.python.org/pypi/virtualenv}{Virtualenv}的安装分为以下几步:

\begin{itemize}
\item \textcolor{etc}{Install pip and Virtualenv.}
\item \textcolor{etc}{Create a Virtualenv environment.}
\item \textcolor{etc}{Activate the Virtualenv environment and install TensorFlow in it.}
\item \textcolor{etc}{After the install you will activate the Virtualenv environment each time you want to use TensorFlow.}
\item 安装 pip 和 Virtualenv.
\item 建立一个 Virtualenv 环境.
\item 激活这个 Virtualenv 环境,并且在此环境下安装 TensorFlow.
\item 安装完成之后,每次你需要使用 TensorFlow 之前必须激活这个 Virtualenv 环境.
\end{itemize}

Ⓔ \textcolor{etc}{Install pip and Virtualenv:}

Ⓒ 安装 pip 和 Virtualenv:

\begin{lstlisting}
# Ubuntu/Linux 64-bit
$ sudo apt-get install python-pip python-dev python-virtualenv
\end{lstlisting}

\begin{lstlisting}
# Mac OS X
$ sudo easy_install pip
$ sudo pip install --upgrade virtualenv
\end{lstlisting}

Ⓔ \textcolor{etc}{Create a Virtualenv environment in the directory} \lstinline{~/tensorflow}:

Ⓒ 在\lstinline{~/tensorflow}路径下建立一个 Virtualenv 环境:

\begin{lstlisting}
$ virtualenv --system-site-packages ~/tensorflow
\end{lstlisting}

Ⓔ \textcolor{etc}{Activate the environment and use pip to install TensorFlow inside it:}

Ⓒ 激活 Virtualenv 环境并使用pip在该环境下安装TensorFlow:

\begin{lstlisting}
$ source ~/tensorflow/bin/activate  # If using bash
$ source ~/tensorflow/bin/activate.csh  # If using csh
(tensorflow)$  # Your prompt should change

# Ubuntu/Linux 64-bit, CPU only:
(tensorflow)$ pip install --upgrade https://storage.googleapis.com/tensorflow/linux/cpu/tensorflow-0.5.0-cp27-none-linux_x86_64.whl

# Ubuntu/Linux 64-bit, GPU enabled:
(tensorflow)$ pip install --upgrade https://storage.googleapis.com/tensorflow/linux/gpu/tensorflow-0.5.0-cp27-none-linux_x86_64.whl

# Mac OS X, CPU only:
(tensorflow)$ pip install --upgrade https://storage.googleapis.com/tensorflow/mac/tensorflow-0.5.0-py2-none-any.whl
\end{lstlisting}

Ⓔ and again for python3:

\begin{lstlisting}
$ source ~/tensorflow/bin/activate  # If using bash
$ source ~/tensorflow/bin/activate.csh  # If using csh
(tensorflow)$  # Your prompt should change

# Ubuntu/Linux 64-bit, CPU only:
(tensorflow)$ pip install --upgrade https://storage.googleapis.com/tensorflow/linux/cpu/tensorflow-0.6.0-cp34-none-linux_x86_64.whl

# Ubuntu/Linux 64-bit, GPU enabled:
(tensorflow)$ pip install --upgrade https://storage.googleapis.com/tensorflow/linux/gpu/tensorflow-0.6.0-cp34-none-linux_x86_64.whl

# Mac OS X, CPU only:
(tensorflow)$ pip3 install --upgrade https://storage.googleapis.com/tensorflow/mac/tensorflow-0.6.0-py3-none-any.whl
\end{lstlisting}

Ⓔ \textcolor{etc}{With the Virtualenv environment activated, you can now \hyperref[test_install]{test your installation}.}

Ⓒ 在 Virtualenv 环境被激活时,您可以\hyperref[test_install]{测试安装}.

Ⓔ \textcolor{etc}{When you are done using TensorFlow, deactivate the environment.}

Ⓒ 当您无需使用 TensorFlow 时,取消激活该环境.

\begin{lstlisting}
(tensorflow)$ deactivate
$  # Your prompt should change back
\end{lstlisting}

Ⓔ \textcolor{etc}{To use TensorFlow later you will have to activate the Virtualenv environment again:}

Ⓒ 如果需要再次使用 TensorFlow 您需要先再次激活 Virtualenv 环境:

\begin{lstlisting}
$ source ~/tensorflow/bin/activate  # If using bash.
$ source ~/tensorflow/bin/activate.csh  # If using csh.
(tensorflow)$  # Your prompt should change.
# Run Python programs that use TensorFlow.
...
# When you are done using TensorFlow, deactivate the environment.
(tensorflow)$ deactivate
\end{lstlisting}


%% Add Anaconda installation

\subsection {Anaconda installation}

\href{https://www.continuum.io/why-anaconda}{Anaconda} is a Python distribution that includes a large number of standard numeric and scientific computing packages. Anaconda uses a package manager called \href{http://conda.pydata.org/}{``conda''} that has its own \href{http://conda.pydata.org/docs/using/envs.html}{environment system} similar to Virtualenv.

\index{Anaconda}  \index{conda}

As with Virtualenv, conda environments keep the dependencies required by different Python projects in separate places. The Anaconda environment installation of TensorFlow will not override pre-existing version of the Python packages needed by TensorFlow.

\begin{itemize}

\item Install Anaconda.

\item Create a conda environment.

\item Activate the conda environment and install TensorFlow in it.

\item After the install you will activate the conda environment each time you want to use TensorFlow.

\end{itemize}

Install Anaconda:

Follow the instructions on the \href{https://www.continuum.io/downloads}{Anaconda download site}

Create a conda environment called \li{tensorflow}:

\begin{lstlisting}

# Python 2.7
$ conda create -n tensorflow python=2.7

# Python 3.4
$ conda create -n tensorflow python=3.4

# Python 3.5
$ conda create -n tensorflow python=3.5

\end{lstlisting}

Activate the environment and use conda or pip to install TensorFlow inside it.

\subsubsection {Using conda}

A community maintained conda package is available from \href{https://github.com/conda-forge/tensorflow-feedstock}{conda-forge}.

Only the CPU version of TensorFlow is available at the moment and can be installed in the conda environment for Python 2 or Python 3.

\begin{lstlisting}
$ source activate tensorflow
(tensorflow)$  # Your prompt should change

# Linux/Mac OS X, Python 2.7/3.4/3.5, CPU only:
(tensorflow)$ conda install -c conda-forge tensorflow
\end{lstlisting}


\subsubsection {Using pip}



\subsubsection {Usage}




%%%%%%%%%%%%%%%%
% Done to here %
%%%%%%%%%%%%%%%%



%
%%
\subsection {Docker Installation} \label{docker_install}
\href{http://docker.com/}{Docker} is a system to build self contained versions of a Linux operating system running on your machine. When you install and run TensorFlow via Docker it completely isolates the installation from pre-existing packages on your machine.

We provide 4 Docker images:

\begin{itemize}
\item \lstinline{b.gcr.io/tensorflow/tensorflow}: TensorFlow CPU binary image.
\item \lstinline{b.gcr.io/tensorflow/tensorflow:latest-devel}:CPU Binary image plus source code.
\item \lstinline{b.gcr.io/tensorflow/tensorflow:latest-gpu}:TensorFlow GPU binary image.
\item \lstinline{b.gcr.io/tensorflow/tensorflow:latest-devel-gpu}:GPU Binary image plus source code.
\end{itemize}

We also have tags with latest replaced by a released version (eg \lstinline{0.6.0-gpu}).

With Docker the installation is as follows:

\begin{itemize}
\item Install Docker on your machine.
\item Create a \href{http://docs.docker.com/engine/installation/ubuntulinux/#create-a-docker-group}{Docker group} to allow launching containers without sudo.
\item Launch a Docker container with the TensorFlow image. The image gets downloaded automatically on first launch.
\end{itemize}

See \href{http://docs.docker.com/engine/installation/}{installing Docker} for instructions on installing Docker on your machine.

After Docker is installed, launch a Docker container with the TensorFlow binary image as follows.

\begin{lstlisting}
$ docker run -it b.gcr.io/tensorflow/tensorflow
\end{lstlisting}

If you're using a container with GPU support, some additional flags must be passed to expose the GPU device to the container. For the default config, we include a \href{https://github.com/tensorflow/tensorflow/blob/master/tensorflow/tools/docker/docker_run_gpu.sh}{script} in the repo with these flags, so the command-line would look like:

\begin{lstlisting}
$ path/to/repo/tensorflow/tools/docker/docker_run_gpu.sh b.gcr.io/tensorflow/tensorflow:gpu
\end{lstlisting}

You can now \hyperref[test_install]{test your installation} within the Docker container.


%
%%
\subsection {Test the TensorFlow installation  |  测试 TensorFlow 安装} \label{test_install}

%%%
\subsubsection {(Optional, Linux) Enable GPU Support}

\textcolor{etc}{If you installed the GPU version of TensorFlow, you must also install the Cuda Toolkit~7.0 and CUDNN~6.5~V2. Please see \hyperref[install_cuda]{Cuda installation}.}

Ⓒ 如果您安装了GPU版本的TensorFlow, 您还需要安装 Cuda Toolkit~7.0 和 CUDNN~6.5~V2.请参阅\hyperref[install_cuda]{Cuda 安装}.

\textcolor{etc}{You also need to set the \lstinline{LD_LIBRARY_PATH} and \lstinline{CUDA_HOME} environment variables. Consider adding the commands below to your} \lstinline{~/.bash_profile}. \textcolor{etc}{These assume your CUDA installation is in \lstinline{/usr/local/cuda}:}

Ⓒ 您需要在先环境变量中设置\lstinline{LD_LIBRARY_PATH} 和 \lstinline{CUDA_HOME}.您可以在\lstinline{~/.bash_profile}中追加一下命令,假设您的CUDA安装位置为\lstinline{/usr/local/cuda}:

\begin{lstlisting}
export LD_LIBRARY_PATH="$LD_LIBRARY_PATH:/usr/local/cuda/lib64"
export CUDA_HOME=/usr/local/cuda
\end{lstlisting}

%%%
\subsubsection {Run TensorFlow from the Command Line  |  从命令行运行TensorFlow}

See \hyperref[comm_prob]{common problems} if an error happens.

Ⓒ 如果遇到任何报错,请参考\hyperref[comm_prob]{常见问题}.

Open a terminal and type the following:

Ⓒ 打开终端,输入以下指令:

\begin{lstlisting}
$ python
...
>>> import tensorflow as tf
>>> hello = tf.constant('Hello, TensorFlow!')
>>> sess = tf.Session()
>>> print(sess.run(hello))
Hello, TensorFlow!
>>> a = tf.constant(10)
>>> b = tf.constant(32)
>>> print(sess.run(a + b))
42
>>>
\end{lstlisting}

%%%
\subsubsection {Run a TensorFlow demo model  |  运行一个TensorFlow的演示模型}

All TensorFlow packages, including the demo models, are installed in the Python library. The exact location of the Python library depends on your system, but is usually one of:

Ⓒ 所有版本的TensorFlow的Python库中包都附带了一些演示模型. 具体位位置取决于您的系统,它们通常会在以下位置出现:

\begin{lstlisting}
/usr/local/lib/python2.7/dist-packages/tensorflow
/usr/local/lib/python2.7/site-packages/tensorflow
\end{lstlisting}

You can find out the directory with the following command:
Ⓒ 您可以用以下指令找到它的路径:

\begin{lstlisting}
$ python -c 'import site; print("\n".join(site.getsitepackages()))'
\end{lstlisting}

The simple demo model for classifying handwritten digits from the MNIST dataset is in the sub-directory \lstinline{models/image/mnist/convolutional.py}. You can run it from the command line as follows:

Ⓒ 在子目录\lstinline{models/image/mnist/convolutional.py}可以找到一个使用MNIST数据集进行手写数字识别的简单案例.您可以使用以下指令在命令行中直接运行:

\begin{lstlisting}
# Using 'python -m' to find the program in the python search path:
$ python -m tensorflow.models.image.mnist.convolutional
Extracting data/train-images-idx3-ubyte.gz
Extracting data/train-labels-idx1-ubyte.gz
Extracting data/t10k-images-idx3-ubyte.gz
Extracting data/t10k-labels-idx1-ubyte.gz
...etc...

# You can alternatively pass the path to the model program file to the python interpreter.
$ python /usr/local/lib/python2.7/dist-packages/tensorflow/models/image/mnist/convolutional.py
...
\end{lstlisting}


%
%%
\subsection {Installing from source}

Ⓔ \textcolor{etc}{When installing from source you will build a pip wheel that you then install using pip. You'll need pip for that, so install it as described \hyperref[pip_install]{above}.}

%%%
\subsubsection {Clone the TensorFlow repository}
\begin{lstlisting}
$ git clone --recurse-submodules https://github.com/tensorflow/tensorflow
\end{lstlisting}
\lstinline{--recurse-submodules} is required to fetch the protobuf library that TensorFlow depends on.


%%%
\subsubsection {Installation for Linux}

\paragraph{Install Bazel}

Follow instructions here to install the dependencies for Bazel. Then download bazel version 0.1.1 using the installer for your system and run the installer as mentioned there:

\begin{lstlisting}
$ chmod +x PATH_TO_INSTALL.SH
$ ./PATH_TO_INSTALL.SH --user
\end{lstlisting}

Remember to replace \lstinline{PATH_TO_INSTALL.SH} with the location where you downloaded the installer.

Finally, follow the instructions in that script to place bazel into your binary path.

\paragraph{Install other dependencies}

\begin{lstlisting}
$ sudo apt-get install python-numpy swig python-dev
\end{lstlisting}

\paragraph{Configure the installation}

Run the configure script at the root of the tree. The configure script asks you for the path to your python interpreter and allows (optional) configuration of the CUDA libraries (see \hyperref[install_cuda]{below}).

This step is used to locate the python and numpy header files.

\begin{lstlisting}
$ ./configure
Please specify the location of python. [Default is /usr/bin/python]:
\end{lstlisting}

\paragraph{Optional: Install CUDA (GPUs on Linux)} \label{install_cuda}

In order to build or run TensorFlow with GPU support, both Cuda Toolkit 7.0 and CUDNN 6.5 V2 from NVIDIA need to be installed.

TensorFlow GPU support requires having a GPU card with NVidia Compute Capability >= 3.5. Supported cards include but are not limited to:

\begin{itemize}
\item NVidia Titan
\item NVidia Titan X
\item NVidia K20
\item NVidia K40
\end{itemize}

Download and install Cuda Toolkit 7.0

https://developer.nvidia.com/cuda-toolkit-70

Install the toolkit into e.g. /usr/local/cuda

Download and install CUDNN Toolkit 6.5

https://developer.nvidia.com/rdp/cudnn-archive

Uncompress and copy the cudnn files into the toolkit directory. Assuming the toolkit is installed in /usr/local/cuda:

\begin{lstlisting}
tar xvzf cudnn-6.5-linux-x64-v2.tgz
sudo cp cudnn-6.5-linux-x64-v2/cudnn.h /usr/local/cuda/include
sudo cp cudnn-6.5-linux-x64-v2/libcudnn* /usr/local/cuda/lib64
\end{lstlisting}

Configure TensorFlow's canonical view of Cuda libraries

When running the configure script from the root of your source tree, select the option Y when asked to build TensorFlow with GPU support.

\begin{lstlisting}
$ ./configure
Please specify the location of python. [Default is /usr/bin/python]:
Do you wish to build TensorFlow with GPU support? [y/N] y
GPU support will be enabled for TensorFlow

Please specify the location where CUDA 7.0 toolkit is installed. Refer to
README.md for more details. [default is: /usr/local/cuda]: /usr/local/cuda

Please specify the location where CUDNN 6.5 V2 library is installed. Refer to
README.md for more details. [default is: /usr/local/cuda]: /usr/local/cuda

Setting up Cuda include
Setting up Cuda lib64
Setting up Cuda bin
Setting up Cuda nvvm
Configuration finished
\end{lstlisting}

This creates a canonical set of symbolic links to the Cuda libraries on your system. Every time you change the Cuda library paths you need to run this step again before you invoke the bazel build command.

Build your target with GPU support

From the root of your source tree, run:

\begin{lstlisting}
$ bazel build -c opt --config=cuda //tensorflow/cc:tutorials_example_trainer

$ bazel-bin/tensorflow/cc/tutorials_example_trainer --use_gpu
# Lots of output. This tutorial iteratively calculates the major eigenvalue of
# a 2x2 matrix, on GPU. The last few lines look like this.
000009/000005 lambda = 2.000000 x = [0.894427 -0.447214] y = [1.788854 -0.894427]
000006/000001 lambda = 2.000000 x = [0.894427 -0.447214] y = [1.788854 -0.894427]
000009/000009 lambda = 2.000000 x = [0.894427 -0.447214] y = [1.788854 -0.894427]
\end{lstlisting}

Note that "--config=cuda" is needed to enable the GPU support.

Enabling Cuda 3.0

TensorFlow officially supports Cuda devices with 3.5 and 5.2 compute capabilities. In order to enable earlier Cuda devices such as Grid K520, you need to target Cuda 3.0. This can be done through TensorFlow unofficial settings with "configure".

\begin{lstlisting}
$ TF_UNOFFICIAL_SETTING=1 ./configure

# Same as the official settings above

WARNING: You are configuring unofficial settings in TensorFlow. Because some
external libraries are not backward compatible, these settings are largely
untested and unsupported.

Please specify a list of comma-separated Cuda compute capabilities you want to
build with. You can find the compute capability of your device at:
https://developer.nvidia.com/cuda-gpus.
Please note that each additional compute capability significantly increases
your build time and binary size. [Default is: "3.5,5.2"]: 3.0

Setting up Cuda include
Setting up Cuda lib64
Setting up Cuda bin
Setting up Cuda nvvm
Configuration finished
\end{lstlisting}

Known issues

Although it is possible to build both Cuda and non-Cuda configs under the same source tree, we recommend to run \"bazel clean\" when switching between these two configs in the same source tree.

You have to run configure before running bazel build. Otherwise, the build will fail with a clear error message. In the future, we might consider making this more conveninent by including the configure step in our build process, given necessary bazel new feature support.


%%%
\subsubsection {Installation for Mac OS X}

We recommend using \href{http://brew.sh/}{homebrew} to install the bazel and SWIG dependencies, and installing python dependencies using easy_install or pip.

\paragraph{Dependencies}

Follow instructions here to install the dependencies for Bazel. You can then use homebrew to install bazel and SWIG:

\begin{lstlisting}
$ brew install bazel swig
\end{lstlisting}

You can install the python dependencies using easy_install or pip. Using easy_install, run

\begin{lstlisting}
$ sudo easy_install -U six
$ sudo easy_install -U numpy
$ sudo easy_install wheel
\end{lstlisting}

We also recommend the \href{https://ipython.org/}{ipython} enhanced python shell, so best install that too:

\begin{lstlisting}
$ sudo easy_install ipython
\end{lstlisting}

\paragraph{Configure the installation}

Run the \lstinline{configure} script at the root of the tree. The configure script asks you for the path to your python interpreter.

This step is used to locate the python and numpy header files.

\begin{lstlisting}
$ ./configure
Please specify the location of python. [Default is /usr/bin/python]:
Do you wish to build TensorFlow with GPU support? [y/N]
\end{lstlisting}

%%%
\subsubsection {Create the pip package and install}

\begin{lstlisting}
$ bazel build -c opt //tensorflow/tools/pip_package:build_pip_package

# To build with GPU support:
$ bazel build -c opt --config=cuda //tensorflow/tools/pip_package:build_pip_package

$ bazel-bin/tensorflow/tools/pip_package/build_pip_package /tmp/tensorflow_pkg

# The name of the .whl file will depend on your platform.
$ pip install /tmp/tensorflow_pkg/tensorflow-0.5.0-cp27-none-linux_x86_64.whl
\end{lstlisting}

%
%%
\subsection {Train your first TensorFlow neural net model  |  训练第一个TensorFlow模型}

Ⓔ \textcolor{etc}{Starting from the root of your source tree, run:}

从根目录开始运行一下指令:

\begin{lstlisting}
$ cd tensorflow/models/image/mnist
$ python convolutional.py
Succesfully downloaded train-images-idx3-ubyte.gz 9912422 bytes.
Succesfully downloaded train-labels-idx1-ubyte.gz 28881 bytes.
Succesfully downloaded t10k-images-idx3-ubyte.gz 1648877 bytes.
Succesfully downloaded t10k-labels-idx1-ubyte.gz 4542 bytes.
Extracting data/train-images-idx3-ubyte.gz
Extracting data/train-labels-idx1-ubyte.gz
Extracting data/t10k-images-idx3-ubyte.gz
Extracting data/t10k-labels-idx1-ubyte.gz
Initialized!
Epoch 0.00
Minibatch loss: 12.054, learning rate: 0.010000
Minibatch error: 90.6%
Validation error: 84.6%
Epoch 0.12
Minibatch loss: 3.285, learning rate: 0.010000
Minibatch error: 6.2%
Validation error: 7.0%
...
...
\end{lstlisting}


%
%%

% \subsection {Common Problems} \label{comm_prob}
\subsection {Common Problems  |  常见问题} \label{comm_prob}

%%%
\subsubsection {GPU-related issues  |  GPU有关问题}

If you encounter the following when trying to run a TensorFlow program:

\begin{lstlisting}
ImportError: libcudart.so.7.0: cannot open shared object file: No such file or directory
\end{lstlisting}

Make sure you followed the the GPU installation \hyperref[install_cuda]{instructions}.

%%%
\subsubsection {Pip installation issues  |  Pip安装中的问题}

\paragraph{\lstinline{Can't find setup.py}}

If, during pip install, you encounter an error like:

\begin{lstlisting}
...
IOError: [Errno 2] No such file or directory: '/tmp/pip-o6Tpui-build/setup.py'
\end{lstlisting}

Solution: upgrade your version of pip:

\begin{lstlisting}
pip install --upgrade pip
\end{lstlisting}

This may require sudo, depending on how pip is installed.

\paragraph{\lstinline{SSLError: SSL_VERIFY_FAILED}}

If, during pip install from a URL, you encounter an error like:

\begin{lstlisting}
...
SSLError: [SSL: CERTIFICATE_VERIFY_FAILED] certificate verify failed
\end{lstlisting}

Solution: Download the wheel manually via curl or wget, and pip install locally.

%%%
\subsubsection {Linux issues}

If you encounter:

\begin{lstlisting}
...
 "__add__", "__radd__",
             ^
SyntaxError: invalid syntax
\end{lstlisting}

Solution: make sure you are using Python 2.7.

%%%
\subsubsection {Mac OS X: ImportError: No module named copyreg}

On Mac OS X, you may encounter the following when importing tensorflow.

\begin{lstlisting}
>>> import tensorflow as tf
...
ImportError: No module named copyreg
\end{lstlisting}

Solution: TensorFlow depends on protobuf, which requires the Python package \lstinline{six-1.10.0}. Apple's default Python installation only provides \lstinline{six-1.4.1}.

You can resolve the issue in one of the following ways:
\begin{itemize}
\item pgrade the Python installation with the current version of \lstinline{six}:
\begin{lstlisting}
$ sudo easy_install -U six
\end{lstlisting}

\item Install TensorFlow with a separate Python library:
  \begin{itemize}
  \item Virtualenv
  \item Docker
  \end{itemize}
Install a separate copy of Python via Homebrew or MacPorts and re-install TensorFlow in that copy of Python.
\end{itemize}




%%%
\subsubsection {Mac OS X: TypeError: \lstinline{__init__()} got an unexpected keyword argument 'syntax'}

On Mac OS X, you may encounter the following when importing tensorflow.

\begin{lstlisting}
>>> import tensorflow as tf
Traceback (most recent call last):
  File "<stdin>", line 1, in <module>
  File "/usr/local/lib/python2.7/site-packages/tensorflow/__init__.py", line 4, in <module>
    from tensorflow.python import *
  File "/usr/local/lib/python2.7/site-packages/tensorflow/python/__init__.py", line 13, in <module>
    from tensorflow.core.framework.graph_pb2 import *
...
  File "/usr/local/lib/python2.7/site-packages/tensorflow/core/framework/tensor_shape_pb2.py", line 22, in <module>
    serialized_pb=_b('\n,tensorflow/core/framework/tensor_shape.proto\x12\ntensorflow\"d\n\x10TensorShapeProto\x12-\n\x03\x64im\x18\x02 \x03(\x0b\x32 .tensorflow.TensorShapeProto.Dim\x1a!\n\x03\x44im\x12\x0c\n\x04size\x18\x01 \x01(\x03\x12\x0c\n\x04name\x18\x02 \x01(\tb\x06proto3')
TypeError: __init__() got an unexpected keyword argument 'syntax'
\end{lstlisting}

This is due to a conflict between protobuf versions (we require protobuf 3.0.0). The best current solution is to make sure older versions of protobuf are not installed, such as:

\begin{lstlisting}
$ pip install --upgrade protobuf
\end{lstlisting}


原文:\href{http://tensorflow.org/get_started/os_setup.md}{Download and Setup}

%!TEX program = xelatex
% Encoding: UTF8
% SEIKA 2016 | seika@live.ca


% Chapter 1
% Section 1.3 Introduction

\section{Basic Usage   ||   使用基础} \label{basic_usage}

Ⓔ \cet{To use TensorFlow you need to understand how TensorFlow:}

\begin{itemize}
\item \cet{Represents computations as  {\em graphs}.}
\item \cet{Executes graphs in the context of  \li{Sessions}.}
\item \cet{Represents data as  {\em tensors}.}
\item \cet{Maintains state with  \li{Variables}.}
\item \cet{Uses feeds and fetches to get data into and out of arbitrary operations.}
\end{itemize}

Ⓒ 使用TensorFlow之前你需要了解关于TensorFlow的以下基础知识:

\begin{itemize}
\item 使用 {\em 图} ({\em graphs}) 来表示计算.
\item 在 {\em 会话} (\li{Session}) 中执行图.
\item 使用 {\em 张量} ({\em tensors}) 来代表数据.
\item 通过 {\em 变量} (\li{Variables}) 维护状态.
\item 使用 {\em 供给} (\li{feeds}) 和 {\em 取回}(\li{fetches})将数据传入或传出任何操作.
\end{itemize}

\index{graph}  \index{session}
\index{tensor}  \index{variable}


%
%%
\subsection{Overview  |  总览}

Ⓔ \cet{TensorFlow is a programming system in which you represent computations as graphs. Nodes in the graph are called ops (short for operations). An op takes zero or more  {\em Tensors}, performs some computation, and produces zero or more  {\em Tensors}. A Tensor is a typed multi-dimensional array. For example, you can represent a mini-batch of images as a 4-D array of floating point numbers with dimensions  \li{[batch, height, width, channels]}.}

Ⓒ TensorFlow 是一个以{\em 图}({\em graphs})来表示计算的编程系统,图中的节点被称之为 op~(operation的缩写). 一个 op 获得零或多个 {\em 张量}( {\em tensors})执行计算,产生零或多个 {\em 张量}。   {\em 张量} 是一个按类型划分的多维数组。  例如, 你可以将一小组图像集表示为一个四维浮点数数组,这四个维度分别是 \li{[batch, height, width, channels]}。

Ⓔ \cet{A TensorFlow graph is a description of computations. To compute anything, a graph must be launched in a  \li{Session}. A Session places the graph ops onto  \li{Devices}, such as CPUs or GPUs, and provides methods to execute them. These methods return tensors produced by ops as \href{http://www.numpy.org/}{numpy} ndarray objects in Python, and as  \li{tensorflow::Tensor} instances in C and C++.}

Ⓒ TensorFlow 中的图是对计算的抽象描述。    在计算开始前, 图必须在 {\em 会话}( \li{Session()}) 中被启动。    {\em 会话}将图的 op 分发到如 CPU 或 GPU 之类的 {\em 设备}( \li{Devices()}) 上, 同时提供执行 op 的方法。    这些方法执行后, 将产生的 {\em 张量}( {\em tensor})返回。    在 Python 语言中, 将返回 \href{http://www.numpy.org}{numpy}的 \li{ndarray} 对象;  在 C 和 C++ 语言中, 将返回 \li{tensorflow::Tensor}实例。


%
%%
%✠ \subsection{The computation graph}
\subsection {The computation graph  |  计算图}
\label{computation_graph}

Ⓔ \cet{TensorFlow programs are usually structured into a  {\em construction phase}, that assembles a graph, and an  {\em execution phase} that uses a session to execute ops in the graph.}

Ⓒ TensorFlow 编程通常可分两个阶段组织: {\em 构建阶段}和 {\em 执行阶段};前者用于组织 {\em 计算图},而后者利用session中执行 {\em 计算图}中的op操作。

Ⓔ \cet{For example, it is common to create a graph to represent and train a neural network in the construction phase, and then repeatedly execute a set of training ops in the graph in the execution phase.}

Ⓒ 例如,在构建阶段创建一个图来表示和训练神经网络,然后在执行阶段反复执行一组op来实现图中的训练。

Ⓔ \cet{TensorFlow can be used from C, C++, and Python programs. It is presently much easier to use the Python library to assemble graphs, as it provides a large set of helper functions not available in the C and C++ libraries.}

Ⓒ TensorFlow 支持 C、 C++、 Python 编程语言。   目前, TensorFlow 的 Python 库更加易用,它提供了大量的辅助函数来简化构建图的工作, 而这些函数在 C 和 C++ 库中尚不被支持。

Ⓔ \cet{The session libraries have equivalent functionalities for the three languages.}

Ⓒ 这三种语言的会话库 (session libraries) 是一致的.

%%%
% \subsubsection {Building the graph}
\subsubsection {Building the graph  |  构建计算图}

Ⓔ \cet{To build a graph start with ops that do not need any input (source ops), such as Constant, and pass their output to other ops that do computation.}

Ⓒ 刚开始基于op建立图的时候一般不需要任何的输入源(source op),例如输入常量( \li{Constance}),再将它们传递给其它 op 执行运算。

Ⓔ \cet{The ops constructors in the Python library return objects that stand for the output of the constructed ops. You can pass these to other ops constructors to use as inputs.}

Ⓒ Python库中的op构造函数返回代表已被组织好的op作为输出对象,这些对象可以传递给其它
op构造函数作为输入。

Ⓔ \cet{The TensorFlow Python library has a default graph to which ops constructors add nodes. The default graph is sufficient for many applications. See the \hyperref[class-tf.graph]{Graph class} documentation for how to explicitly manage multiple graphs.}

Ⓒ TensorFlow Python 库有一个可被 op构造函数加入计算结点的默认图 (default graph)。   对大多数应用来说,这个默认图已经足够用了。  阅读 \hyperref[class-tf.graph]{Graph 类}文档来了解如何明晰的管理多个图.

\begin{lstlisting}
import tensorflow as tf

# Create a Constant op that produces a 1x2 matrix.  The op is
# added as a node to the default graph.
#
# The value returned by the constructor represents the output
# of the Constant op.
matrix1 = tf.constant([[3., 3.]])

# Create another Constant that produces a 2x1 matrix.
matrix2 = tf.constant([[2.],[2.]])

# Create a Matmul op that takes 'matrix1' and 'matrix2' as inputs.
# The returned value, 'product', represents the result of the matrix
# multiplication.
product = tf.matmul(matrix1, matrix2)
\end{lstlisting}

Ⓔ \cet{The default graph now has three nodes: two constant() ops and one matmul() op. To actually multiply the matrices, and get the result of the multiplication, you must launch the graph in a session.}

Ⓒ 默认图现在拥有三个节点,两个 \li{constant()} op,一个 \li{matmul()} op. 为了真正进行矩阵乘法运算,得到乘法结果, 你必须在一个会话(session)中载入动这个图。


%%%
\subsubsection {Launching the graph in a session  |  在会话中载入图} \label{launching_graph}

Ⓔ \cet{Launching follows construction. To launch a graph, create a Session object. Without arguments the session constructor launches the default graph.}

Ⓔ \cet{See the \hyperref[class-tf.session]{Session class} for the complete session API.}

Ⓒ 构建过程完成后就可运行执行过程。  为了载入之前所构建的图,必须先创建一个 {\em 会话} (\li{Session}) 对象。  会话构建器在未指明参数时会载入默认的图。

Ⓒ 完整的会话API资料,请参见\hyperref[class-tf.session]{ {\em 会话类}(Session object)}。

\begin{lstlisting}
# Launch the default graph.
sess = tf.Session()

# To run the matmul op we call the session 'run()' method, passing 'product'
# which represents the output of the matmul op.  This indicates to the call
# that we want to get the output of the matmul op back.
#
# All inputs needed by the op are run automatically by the session.  They
# typically are run in parallel.
#
# The call 'run(product)' thus causes the execution of threes ops in the
# graph: the two constants and matmul.
#
# The output of the op is returned in 'result' as a numpy `ndarray` object.
result = sess.run(product)
print(result)
# ==> [[ 12.]]

# Close the Session when we're done.
sess.close()
\end{lstlisting}

Ⓔ \cet{Sessions should be closed to release resources. You can also enter a Session with a "with" block. The Session closes automatically at the end of the with block.}

Ⓒ 会话在完成后必须关闭以释放资源。  你也可以使用 \li{"with"}句块开始一个会话,该会话将在 \li{"with"}句块结束时自动关闭。

\begin{lstlisting}
with tf.Session() as sess:
  result = sess.run([product])
  print(result)
\end{lstlisting}

Ⓔ \cet{The TensorFlow implementation translates the graph definition into executable operations distributed across available compute resources, such as the CPU or one of your computer's GPU cards. In general you do not have to specify CPUs or GPUs explicitly. TensorFlow uses your first GPU, if you have one, for as many operations as possible.}

Ⓒ TensorFlow 事实上通过一个“翻译”过程,将定义的图转化为不同的可用计算资源间实现分布计算的操作,如CPU或是显卡GPU。  通常不需要用户指定具体使用的CPU或GPU,TensorFlow 能自动检测并尽可能的充分利用找到的第一个 GPU 进行运算。

Ⓔ \cet{If you have more than one GPU available on your machine, to use a GPU beyond the first you must assign ops to it explicitly. Use with...Device statements to specify which CPU or GPU to use for operations:}

Ⓒ 如果你的设备上有不止一个GPU,你需要明确指定op操作到不同的运算设备以调用它们。  使用 \li{with...Device}语句明确指定哪个CPU或GPU将被调用:

\begin{lstlisting}
with tf.Session() as sess:
  with tf.device("/gpu:1"):
    matrix1 = tf.constant([[3., 3.]])
    matrix2 = tf.constant([[2.],[2.]])
    product = tf.matmul(matrix1, matrix2)
    ...
\end{lstlisting}

Ⓔ \cet{Devices are specified with strings. The currently supported devices are:}\\
Ⓔ \cet{ \li{"/cpu:0"}: The CPU of your machine.}\\
Ⓔ \cet{ \li{"/gpu:0"}: The GPU of your machine, if you have one.}\\
Ⓔ \cet{ \li{"/gpu:1"}: The second GPU of your machine, etc.}

Ⓔ \cet{See Using GPUs for more information about GPUs and TensorFlow.}

Ⓒ 使用字符串指定设备,目前支持的设备包括:\\
Ⓒ  \li{"/cpu:0"}:计算机的CPU;\\
Ⓒ  \li{"/gpu:0"}:计算机的第一个GPU,如果可用;\\
Ⓒ  \li{"/gpu:1"}:计算机的第二个GPU,以此类推。

Ⓒ 关于使用GPU的更多信息,请参阅\textbf{GPU使用}。

%
%%
\subsection{Interactive Usage  |  交互式使用}

Ⓔ \cet{The Python examples in the documentation launch the graph with a \hyperref[class-tf.session]{ \li{Session}} and use the \hyperref[tf.session.run]{ \li{Session.run()}} method to execute operations.}

Ⓔ \cet{For ease of use in interactive Python environments, such as \href{http://ipython.org/}{ \li{IPython}} you can instead use the \hyperref[class-tf.interactivesession]{ \li{InteractiveSession}} class, and the \hyperref[tf.tensor.eval]{ \li{Tensor.eval()}} and \hyperref[tf.operation.run]{ \li{Operation.run()}} methods. This avoids having to keep a variable holding the session.}

Ⓒ 文档中的 Python 示例使用一个 \li{Session} 会话来启动计算图, 并调用 \hyperref[tf.session.run]{ \li{Session.run()}} 方法执行操作。

Ⓒ 考虑到如\href{http://ipython.org}{IPython}这样的交互式Python环境的易用, 可以使用\hyperref[class-tf.interactivesession]{ \li{InteractiveSession}} 代替 \li{Session}类, 使用 \hyperref[tf.tensor.eval]{ \li{Tensor.eval()}}和 \hyperref[tf.operation.run]{ \li{Operation.run()}} 方法代替  \li{Session.run()}. 这样可以避免使用一个变量来持有会话.

\begin{lstlisting}
# Enter an interactive TensorFlow Session.
import tensorflow as tf
sess = tf.InteractiveSession()

x = tf.Variable([1.0, 2.0])
a = tf.constant([3.0, 3.0])

# Initialize 'x' using the run() method of its initializer op.
x.initializer.run()

# Add an op to subtract 'a' from 'x'.  Run it and print the result
sub = tf.sub(x, a)
print(sub.eval())
# ==> [-2. -1.]

# Close the Session when we're done.
sess.close()
\end{lstlisting}

%
%%

\subsection{Tensors  |  张量}
Ⓔ \cet{TensorFlow programs use a tensor data structure to represent all data -- only tensors are passed between operations in the computation graph. You can think of a TensorFlow tensor as an n-dimensional array or list. A tensor has a static type, a rank, and a shape. To learn more about how TensorFlow handles these concepts, see the \href{https://www.tensorflow.org/versions/master/resources/dims_types.html#tensor-ranks-shapes-and-types}{Rank, Shape, and Type} reference.}

Ⓒ TensorFlow 程序使用 tensor 数据结构来代表所有的数据, 计算图中, 操作间传递的数据都是 tensor. 你可以把 TensorFlow 的张量看作是一个 n 维的数组或列表. 一个 tensor 包含一个静态类型 rank, 和一个 shape. 想了解 TensorFlow 是如何处理这些概念的, 参见 Rank, Shape, 和 Type。


%
%%
\subsection{Variables  |  变量}

Ⓔ \cet{Variables maintain state across executions of the graph. The following example shows a variable serving as a simple counter. See \hyperref[variables]{Variables} for more details.}

Ⓒ  {\em 变量}维持了图执行过程中的状态信息。  下面的例子演示了如何使用变量实现一个简单的计数器,更多细节详见\hyperref[variables]{变量}章节。

\begin{lstlisting}
# Create a Variable, that will be initialized to the scalar value 0.
# 建立一个变量,用0初始化它的值
state = tf.Variable(0, name="counter")

# Create an Op to add one to `state`.

one = tf.constant(1)
new_value = tf.add(state, one)
update = tf.assign(state, new_value)

# Variables must be initialized by running an `init` Op after having
# launched the graph.  We first have to add the `init` Op to the graph.
init_op = tf.initialize_all_variables()

# Launch the graph and run the ops.
with tf.Session() as sess:
  # Run the 'init' op
  sess.run(init_op)
  # Print the initial value of 'state'
  print(sess.run(state))
  # Run the op that updates 'state' and print 'state'.
  for _ in range(3):
    sess.run(update)
    print(sess.run(state))

# output:

# 0
# 1
# 2
# 3
\end{lstlisting}

Ⓔ \cet{The  \li{assign()} operation in this code is a part of the expression graph just like the  \li{add()} operation, so it does not actually perform the assignment until  \li{run()} executes the expression.}

Ⓒ 代码中 \li{assign()}操作是图所描绘的表达式的一部分, 正如\lstinline {add()}操作一样. 所以在调用\lstinline {run()}执行表达式之前, 它并不会真正执行赋值操作.

Ⓔ \cet{You typically represent the parameters of a statistical model as a set of Variables. For example, you would store the weights for a neural network as a tensor in a Variable. During training you update this tensor by running a training graph repeatedly.}

Ⓒ 通常会将一个统计模型中的参数表示为一组变量. 例如, 你可以将一个神经网络的权重作为某个变量存储在一个 tensor 中. 在训练过程中, 通过重复运行训练图, 更新这个 tensor.


%
%%
\subsection{Fetches  |  取回}

Ⓔ \cet{To fetch the outputs of operations, execute the graph with a  \li{run()} call on the  \li{Session} object and pass in the tensors to retrieve. In the previous example we fetched the single node  \li{state}, but you can also fetch multiple tensors:}

Ⓒ 为了取回操作的输出内容, 可以在使用  \li{Session} 对象的  \li{run()} 调用 执行图时, 传入一些 tensor,这些 tensor 会帮助你取回结果. 在之前的例子里, 我们只取回了单个节点 \li{state}, 但是你也可以取回多个tensor:

\begin{lstlisting}
input1 = tf.constant(3.0)
input2 = tf.constant(2.0)
input3 = tf.constant(5.0)
intermed = tf.add(input2, input3)
mul = tf.mul(input1, intermed)

with tf.Session() as sess:
  result = sess.run([mul, intermed])
  print(result)

# output:
# [array([ 21.], dtype=float32), array([ 7.], dtype=float32)]
\end{lstlisting}

Ⓔ \cet{All the ops needed to produce the values of the requested tensors are run once (not once per requested tensor).}

Ⓒ 需要获取的多个 tensor 值,在 op 的一次运行中一起获得(而不是逐个去获取 tensor)。

%
%%
\subsection{Feeds  |  供给}

Ⓔ \cet{The examples above introduce tensors into the computation graph by storing them in  \li{Constants} and  \li{Variables}. TensorFlow also provides a feed mechanism for patching a tensor directly into any operation in the graph.}

Ⓒ 上述示例在计算图中引入了 tensor, 以  {\em 常量} ( \li{Constants}) 或  {\em 变量}( \li{Variables}) 的形式存储. TensorFlow 还提  {\em 供给}( {\em feed}) 机制, 该机制可临时替代图中的任意操作中的 tensor 可以对图中任何操作提交补丁, 直接插入一个 tensor.

Ⓔ \cet{A feed temporarily replaces the output of an operation with a tensor value. You supply feed data as an argument to a  \li{run()} call. The feed is only used for the run call to which it is passed. The most common use case involves designating specific operations to be "feed" operations by using  \li{tf.placeholder()} to create them:}

Ⓒ feed 使用一个 tensor 值临时替换一个操作的输出结果. 你可以提供 feed 数据作为 \lstinline {run()} 调用的参数.feed 只在调用它的方法内有效, 方法结束, feed 就会消失. 最常见的用例是将某些特殊的操作指定为 "feed" 操作, 标记的方法是使用 \li{tf.placeholder()}为这些操作创建占位符.

\begin{lstlisting}
input1 = tf.placeholder(tf.float32)
input2 = tf.placeholder(tf.float32)
output = tf.mul(input1, input2)

with tf.Session() as sess:
  print(sess.run([output], feed_dict={input1:[7.], input2:[2.]}))

# output:
# [array([ 14.], dtype=float32)]
\end{lstlisting}

Ⓔ \cet{A  \li{placeholder()} operation generates an error if you do not supply a feed for it. See the \hyperref[minist_tf]{MNIST fully-connected feed tutorial} (\href{https://tensorflow.googlesource.com/tensorflow/+/master/tensorflow/g3doc/tutorials/mnist/fully_connected_feed.py}{source code}) for a larger-scale example of feeds.}

Ⓒ 如果没有正确供给,  \li{placeholder()} 操作将会产生一个错误提示.关于feed的规模更大的案例,参见\hyperref[minist_tf]{MNIST 全连通 feed 教程}以及其\href{https://tensorflow.googlesource.com/tensorflow/+/master/tensorflow/g3doc/tutorials/mnist/fully_connected_feed.py}{源代码}。

\href{http://tensorflow.org/get_started/basic_usage.md}{原文:Basic Usage}

%%%% 第二章
\newpage
\chapter{基础教程}
%!TEX program = xelatex
% Encoding: UTF8
% SEIKA 2015


% Chapter 2 TutorialsHow to ...
% Section 2.1

\section*{本章摘要}

\hyperref[MINIST_beginner]{ \cet{\textbf{MNIST For ML Beginners}}  ||  \textbf{MNIST 机器学习入门}}

 \cet{If you're new to machine learning, we recommend starting here. You'll learn about a classic problem, handwritten digit classification (MNIST), and get a gentle introduction to multiclass classification.}

如果你是机器学习领域的新手, 我们推荐你从本文开始阅读. 本文通过讲述一个经典的问题, 手写数字识别 (MNIST), 让你对多类分类 (multiclass classification) 问题有直观的了解.

 \cet{\textbf{Deep MNIST for Experts}}  ||  \textbf{深入MNIST}

 \cet{If you're already familiar with other deep learning software packages, and are already familiar with MNIST, this tutorial with give you a very brief primer on TensorFlow.}

如果你已经对其它深度学习软件比较熟悉, 并且也对 MNIST 很熟悉, 这篇教程能够引导你对 TensorFlow 有初步了解.

\hyperref[MINIST_pros]{View Tutorial | 阅读该教程}

 \cet{\textbf{TensorFlow Mechanics 101}}  ||  \textbf{}

 \cet{This is a technical tutorial, where we walk you through the details of using TensorFlow infrastructure to train models at scale. We use again MNIST as the example.}

这是一篇技术教程, 详细介绍了如何使用 TensorFlow 架构训练大规模模型. 本文继续使用MNIST 作为例子.

\hyperref[tf_mech101]{View Tutorial | 阅读该教程}

\textbf{Convolutional Neural Networks}

An introduction to convolutional neural networks using the CIFAR-10 data set. Convolutional neural nets are particularly tailored to images, since they exploit translation invariance to yield more compact and effective representations of visual content.

这篇文章介绍了如何使用 TensorFlow 在 CIFAR-10 数据集上训练卷积神经网络. 卷积神经网络是为图像识别量身定做的一个模型. 相比其它模型, 该模型利用了平移不变性(translation invariance), 从而能够更更简洁有效地表示视觉内容.

View Tutorial

\textbf{Vector Representations of Words}

This tutorial motivates why it is useful to learn to represent words as vectors (called word embeddings). It introduces the word2vec model as an efficient method for learning embeddings. It also covers the high-level details behind noise-contrastive training methods (the biggest recent advance in training embeddings).

本文让你了解为什么学会使用向量来表示单词, 即单词嵌套 (word embedding), 是一件很有用的事情. 文章中介绍的 word2vec 模型, 是一种高效学习嵌套的方法. 本文还涉及了对比噪声(noise-contrastive) 训练方法的一些高级细节, 该训练方法是训练嵌套领域最近最大的进展.

View Tutorial

\textbf{Recurrent Neural Networks}

An introduction to RNNs, wherein we train an LSTM network to predict the next word in an English sentence. (A task sometimes called language modeling.)

一篇 RNN 的介绍文章, 文章中训练了一个 LSTM 网络来预测一个英文句子的下一个单词(该任务有时候被称作语言建模).

View Tutorial

\textbf{Sequence-to-Sequence Models}

A follow on to the RNN tutorial, where we assemble a sequence-to-sequence model for machine translation. You will learn to build your own English-to-French translator, entirely machine learned, end-to-end.

RNN 教程的后续, 该教程采用序列到序列模型进行机器翻译. 你将学会构建一个完全基于机器学习,端到端的\emph{英语-法语}翻译器.

View Tutorial

\textbf{Mandelbrot Set}

TensorFlow can be used for computation that has nothing to do with machine learning. Here's a naive implementation of Mandelbrot set visualization.

TensorFlow 可以用于与机器学习完全无关的其它计算领域. 这里实现了一个原生的 Mandelbrot 集合的可视化程序.

View Tutorial

\textbf{Partial Differential Equations}

As another example of non-machine learning computation, we offer an example of a naive PDE simulation of raindrops landing on a pond.

这是另外一个非机器学习计算的例子, 我们利用一个原生实现的偏微分方程, 对雨滴落在池塘上的过程进行仿真.

View Tutorial

\textbf{MNIST Data Download}

Details about downloading the MNIST handwritten digits data set. Exciting stuff.

一篇关于下载 MNIST 手写识别数据集的详细教程.

View Tutorial

\textbf{Image Recognition}

How to run object recognition using a convolutional neural network trained on ImageNet Challenge data and label set.

如何利用受过训练的ImageNet挑战数据和标签集卷积神经网络来运行物体识别。

View Tutorial

We will soon be releasing code for training a state-of-the-art Inception model.

Deep Dream Visual Hallucinations

Building on the Inception recognition model, we will release a TensorFlow version of the Deep Dream neural network visual hallucination software.

我们也将公布一个训练高级的Iception模型所用的代码。

COMING SOON
%!TEX program = xelatex
% Encoding: UTF8
% SEIKA 2015


% Chapter 2 TutorialsHow to ...
% Section 2.2

\newpage
\section {MNIST机器学习入门}

\label{MINIST_beginner}

Ⓔ \cet{This tutorial is intended for readers who are new to both machine learning and TensorFlow. If you already know what MNIST is, and what softmax (multinomial logistic) regression is, you might prefer this faster paced tutorial. Be sure to install TensorFlow before starting either tutorial.}

Ⓒ 本教程的目标读者是对机器学习和TensorFlow都不太了解的新手.  如果你已经了解MNIST和softmax回归(softmax regression)的相关知识, 你可以阅读这个快速上手教程.

Ⓔ \cet{When one learns how to program, there's a tradition that the first thing you do is print "Hello World." Just like programming has Hello World, machine learning has MNIST.}

Ⓒ 当我们开始学习编程的时候, 第一件事往往是学习打印 ``Hello World''.  就好比编程入门有Hello World, 机器学习入门有MNIST.

\index{MNIST 数据集}

Ⓔ \cet{MNIST is a simple computer vision dataset. It consists of images of handwritten digits like these:}

Ⓒ MNIST 是一个入门级的计算机视觉数据集, 它包含各种手写数字图片:

\begin{figure}[htbp]
\centering
\includegraphics[width=.55\textwidth]{../SOURCE/images/MNIST.png}
\caption{}
\end{figure}

Ⓔ \cet{It also includes labels for each image, telling us which digit it is. For example, the labels for the above images are 5, 0, 4, and 1.}

Ⓒ 它也包含每一张图片对应的标签, 告诉我们这个是数字几.  比如, 上面这四张图片的标签分别是5,0,4,1.

Ⓔ \cet{In this tutorial, we're going to train a model to look at images and predict what digits they are. Our goal isn't to train a really elaborate model that achieves state-of-the-art performance -- although we'll give you code to do that later! -- but rather to dip a toe into using TensorFlow. As such, we're going to start with a very simple model, called a Softmax Regression.}

Ⓒ 在此教程中, 我们将训练一个机器学习模型用于预测图片里面的数字.  我们的目的不是要设计一个世界一流的复杂模型 -- 尽管我们会在之后给你源代码去实现一流的预测模型 -- 而是要介绍下如何使用 TensorFlow.  所以, 我们这里会从一个很简单的数学模型开始, 它叫做 Softmax Regression.

Ⓔ \cet{The actual code for this tutorial is very short, and all the interesting stuff happens in just three lines. However, it is very important to understand the ideas behind it: both how TensorFlow works and the core machine learning concepts. Because of this, we are going to very carefully work through the code.}

Ⓒ 对应这个教程的实现代码很短, 而且真正有意思的内容只包含在三行代码里面.  但是, 去理解包含在这些代码里面的设计思想是非常重要的:TensorFlow 工作流程和机器学习的基本概念.  因此, 这个教程会很详细地介绍这些代码的实现原理.

% \subsection {The MNIST Data  |  MNIST数据集}
\subsection {MNIST数据集}

Ⓔ \cet{The MNIST data is hosted on \href{http://yann.lecun.com/exdb/mnist/}{Yann LeCun's website}. For your convenience, we've included some python \href{https://tensorflow.googlesource.com/tensorflow/+/master/tensorflow/examples/tutorials/mnist/input_data.py}{code} to download and install the data automatically. You can either download the code and import it as below, or simply copy and paste it in.}

Ⓒ MNIST数据集的官网是\href{http://yann.lecun.com/exdb/mnist/}{Yann LeCun's website}.  在这里, 我们提供了一份python源代码用于自动下载和安装这个数据集.  你可以下载这段\href{https://tensorflow.googlesource.com/tensorflow/+/master/tensorflow/examples/tutorials/mnist/input_data.py}{代码}, 然后用下面的代码导入到你的项目里面, 也可以直接复制粘贴到你的代码文件里面.


%\begin{lstlisting}[language={[ANSI]Python}]
\begin{lstlisting}
import input_data
mnist = input_data.read_data_sets("MNIST_data/", one_hot=True)
\end{lstlisting}

Ⓔ \cet{The downloaded data is split into three parts, 55,000 data points of training data \li{(mnist.train)}, 10,000 points of test data \li{(mnist.test)}, and 5,000 points of validation data \li{(mnist.validation)}. This split is very important: it's essential in machine learning that we have separate data which we don't learn from so that we can make sure that what we've learned actually generalizes!}

Ⓒ 下载下来的数据集可被分为三部分:55000 行训练用点数据集(\li{mnist.train}), 10000 行测试数据集(\li{mnist.test}), 以及5000行验证数据集(\li{mnist.validation}).  这样的切分很重要:在机器学习模型设计时必须有一个单独的测试数据集不用于训练而是用来评估这个模型的性能, 从而更加容易把设计的模型推广到其他数据集上(泛化).

Ⓔ \cet{As mentioned earlier, every MNIST data point has two parts: an image of a handwritten digit and a corresponding label. We will call the images "xs" and the labels "ys". Both the training set and test set contain xs and ys, for example the training images are \li{mnist.train.images} and the train labels are \li{mnist.train.labels}.}

Ⓒ 正如前面提到的一样, 每一个MNIST数据单元有两部分组成:一张包含手写数字的图片和一个对应的标签.  我们把这些图片设为 ``xs'', 把这些标签设为 ``ys''.  训练数据集和测试数据集都包含xs和ys, 比如训练数据集的图片是\li{mnist.train.images} , 训练数据集的标签是\li{mnist.train.labels}.

Ⓔ \cet{Each image is 28 pixels by 28 pixels. We can interpret this as a big array of numbers:}

Ⓒ 每一张图片包含$ 28 \times 28$像素.  我们可以用一个数字数组来表示这张图片:

\begin{figure}[htbp]
\centering
\includegraphics[width=.8\textwidth]{../SOURCE/images/MNIST-Matrix.png}
\caption{}
\end{figure}

Ⓔ \cet{We can flatten this array into a vector of $ 28 \times 28 = 784$ numbers. It doesn't matter how we flatten the array, as long as we're consistent between images. From this perspective, the MNIST images are just a bunch of points in a 784-dimensional vector space, with a \href{http://colah.github.io/posts/2014-10-Visualizing-MNIST/}{very rich structure} (warning: computationally intensive visualizations).}

Ⓒ 我们把这个数组展开成一个向量, 长度是 $ 28 \times 28 = 784$.  如何展开这个数组(数字间的顺序)不重要, 只要保持各个图片采用相同的方式展开.  从这个角度来看, MNIST数据集的图片就是在784维向量空间里面的点, 并且拥有比较\href{http://colah.github.io/posts/2014-10-Visualizing-MNIST/}{复杂的结构} (注意: 此类数据的可视化是计算密集型的).

Ⓔ \cet{Flattening the data throws away information about the 2D structure of the image. Isn't that bad? Well, the best computer vision methods do exploit this structure, and we will in later tutorials. But the simple method we will be using here, a softmax regression, won't.}

Ⓒ 展平图片的数字数组会丢失图片的二维结构信息.  这显然是不理想的, 最优秀的计算机视觉方法会挖掘并利用这些结构信息, 我们会在后续教程中介绍.  但是在这个教程中我们忽略这些结构, 所介绍的简单数学模型, softmax回归(softmax regression), 不会利用这些结构信息.

Ⓔ \cet{The result is that \li{mnist.train.images} is a tensor (an n-dimensional array) with a shape of \li{[55000, 784]}. The first dimension indexes the images and the second dimension indexes the pixels in each image. Each entry in the tensor is the pixel intensity between 0 and 1, for a particular pixel in a particular image.}

Ⓒ 因此, 在MNIST训练数据集中, \li{mnist.train.images}是一个形状为 \li{[55000, 784]} 的张量, 第一个维度数字用来索引图片, 第二个维度数字用来索引每张图片中的像素点.  在此张量里的每一个元素, 都表示某张图片里的某个像素的亮度值, 值介于0和1之间.

\begin{figure}[htbp]
\centering
\includegraphics[width=.65\textwidth]{../SOURCE/images/mnist-train-xs.png}
\caption{}
\end{figure}

Ⓔ \cet{The corresponding labels in MNIST are numbers between 0 and 9, describing which digit a given image is of. For the purposes of this tutorial, we're going to want our labels as "one-hot vectors". A one-hot vector is a vector which is 0 in most dimensions, and 1 in a single dimension. In this case, the $n$th digit will be represented as a vector which is 1 in the $n$th dimensions. For example, 3 would be $[0,0,0,1,0,0,0,0,0,0]$. Consequently, \li{mnist.train.labels} is a \li{[55000, 10]} array of floats.}

Ⓒ 相对应的MNIST数据集的标签是介于0到9的数字, 用来描述给定图片里表示的数字.  为了用于这个教程, 我们使标签数据是 ``one-hot vectors''.  one-hot 向量是指除了某一位的数字是1以外其余各维度数字都是0.  所以在此教程中, 数字 n 将表示成一个只有在第 $n$ 维度(从0开始)数字为 1 的 10 维向量.  比如, 标签0将表示成(\li{[1,0,0,0,0,0,0,0,0,0,0]}).  因此, \li{mnist.train.labels}是一个 \li{[55000, 10]} 的数字矩阵.

\begin{figure}[htbp]
\centering
\includegraphics[width=.7\textwidth]{../SOURCE/images/mnist-train-ys.png}
\caption{}
\end{figure}

Ⓔ \cet{We're now ready to actually make our model!}

Ⓒ 现在, 我们准备开始真正的建模之旅!

\subsection {Softmax回归介绍}

Ⓔ \cet{We know that every image in MNIST is a digit, whether it's a zero or a nine. We want to be able to look at an image and give probabilities for it being each digit. For example, our model might look at a picture of a nine and be 80\% sure it's a nine, but give a 5\% chance to it being an eight (because of the top loop) and a bit of probability to all the others because it isn't sure.}

Ⓒ 我们知道MNIST数据集的每一张图片都表示一个(0到9的)数字.  那么, 如果模型若能看到一张图就能知道它属于各个数字的对应概率就好了。比如, 我们的模型可能看到一张数字"9"的图片, 就判断出它是数字"9"的概率为80\%, 而有5\%的概率属于数字"8"(因为8和9都有上半部分的小圆), 同时给予其他数字对应的小概率(因为该图像代表它们的可能性微乎其微).  \index{Softmax regression}

Ⓔ \cet{This is a classic case where a softmax regression is a natural, simple model. If you want to assign probabilities to an object being one of several different things, softmax is the thing to do. Even later on, when we train more sophisticated models, the final step will be a layer of softmax.}

Ⓒ 这是能够体现softmax回归自然简约的一个典型案例.  softmax模型可以用来给不同的对象分配概率.  在后文, 我们训练更加复杂的模型时, 最后一步也往往需要用softmax来分配概率.

Ⓔ \cet{A softmax regression has two steps: first we add up the evidence of our input being in certain classes, and then we convert that evidence into probabilities.}

Ⓒ softmax回归(softmax regression)分两步:首先对输入被分类对象属于某个类的 ``证据''集合起来, 然后将这些 ``证据''转化为概率.

Ⓔ \cet{To tally up the evidence that a given image is in a particular class, we do a weighted sum of the pixel intensities. The weight is negative if that pixel having a high intensity is evidence against the image being in that class, and positive if it is evidence in favor.}

Ⓒ 我们使用加权的方法来累积 \footnote{这里是指累加输入图片每个像素亮度值与该像素位置权重的乘积,即公式\ref{eq.evidence_i}的前部分} 属于同一类的图片中的 ``证据''.  如果图片的像素强有力的体现该图不属于某个类, 则权重为负数, 相反如果这个像素拥有有利的证据支持这张图片属于这个类, 那么权值为正.

Ⓔ \cet{The following diagram shows the weights one model learned for each of these classes. Red represents negative weights, while blue represents positive weights.}

Ⓒ 下面的图片显示了一个模型学习到的图片上每个像素对于各数字类的权值.  红色代表权值为负, 蓝色代表权值为正.

\begin{figure}[htbp]
\centering
\includegraphics[width=.65\textwidth]{../SOURCE/images/softmax-weights.png}
\caption{}
\end{figure}

Ⓔ \cet{We also add some extra evidence called a bias. Basically, we want to be able to say that some things are more likely independent of the input. The result is that the evidence for a class $i$ given an input $x$ is:}

Ⓒ 我们也要引入额外的 ``证据'', 称之为偏置量 (bias).  总的来说, 我们希望它代表了与所输入向量无关的判断证据.  因此,对于给定的输入图片 $x$ 代表 (属于) 数字 $i$ 的 总体证据 可以表示为:
\footnote{ $i$ index of numbers, 0--9 \\
$j$, index of pixels, 0 -- 784}
\begin{equation}
\label{eq.evidence_i}
evidence_i = \sum_j{W_{i,j}}x_j+b_i
\end{equation}\\
Ⓔ \cet{where $W_i$ is the weights and $b_i$ is the bias for class $i$, and $j$ is an index for summing over the pixels in our input image $x$. We then convert the evidence tallies into our predicted probabilities y using the "softmax" function:}\\
Ⓒ 其中, $W_i$ 代表权重, $b_i$ 代表第 $i$ 类(数字)的偏置量, $j$ 索引了输入图片 $x$ 的所有像素.  之后用 softmax 函数 把这些证据转换成概率 $y$:\\
\begin{equation}
y = softmax(evidence)
\end{equation}\\
Ⓔ \cet{Here softmax is serving as an "activation" or "link" function, shaping the output of our linear function into the form we want -- in this case, a probability distribution over 10 cases. You can think of it as converting tallies of evidence into probabilities of our input being in each class. It's defined as:}\\
Ⓒ 这里的softmax可以看成是一个\emph{激励}(activation)函数或是\emph{链接}(link)函数, 把我们定义的线性函数的输出转换成我们想要的格式, 也就是关于10个数字类的概率分布.  因此, 给定一张图片, 它对于每一个数字的吻合度可以被softmax函数转换成为一个概率值.  softmax函数可以定义为:\\
\begin{equation}
\label{eq.softmax}
softmax(x) = normalize(exp(x))
\end{equation}\\
Ⓔ \cet{If you expand that equation out, you get:}\\
展开等式右边的子式, 可以得到:\\
\begin{equation}
softmax(x)_i = \frac{exp(x_i)}{\sum_j{exp(x_j)}}
\end{equation}\\
Ⓔ But it's often more helpful to think of softmax the first way: exponentiating its inputs and then normalizing them. The exponentiation means that one more unit of evidence increases the weight given to any hypothesis multiplicatively. And conversely, having one less unit of evidence means that a hypothesis gets a fraction of its earlier weight. No hypothesis ever has zero or negative weight. Softmax then normalizes these weights, so that they add up to one, forming a valid probability distribution. (To get more intuition about the softmax function, check out the \href{http://neuralnetworksanddeeplearning.com/chap3.html#softmax}{section} on it in Michael Nieslen's book, complete with an interactive visualization.)\\
Ⓒ 但是更多时候,公式~\ref{eq.softmax} 更易于理解 softmax 模型函: 先对输入值用指数函数\footnote{$e$ 为底数的指数函数}求值, 再归一化这些结果.  这个指数运算表示, 更大的证据对应更大的假设模型(hypothesis)里面的乘数权重值.  反之, 拥有更少的证据意味着在假设模型里面拥有更小的乘数系数.  假设模型里的权值不可以是0值或者负值.  Softmax然后会正则化这些权重值, 使它们的总和等于1, 以此构造一个有效的概率分布.  (更多的关于Softmax函数的信息, 可以参考Michael Nieslen的书里面的\href{http://neuralnetworksanddeeplearning.com/chap3.html#softmax}{这个部分}, 其中有关于softmax的可交互式的可视化解释.  )

Ⓔ You can picture our softmax regression as looking something like the following, although with a lot more $x$s. For each output, we compute a weighted sum of the $x$s, add a bias, and then apply softmax.\\
可以把我们的softmax回归模型想像为下图的结构, 尽管事实上 $x$ 的数量会有很多。 对于每一个输出结果 $y$, 我们需要计算 所有 $x$的加权和,再加上对应的偏置量, 再用 softmax 函数求归一化概率。
\begin{center}
\includegraphics[width=.65\textwidth]{../SOURCE/images/softmax-regression-scalargraph.png}
\end{center}
Ⓔ If we write that out as equations, we get:\\
如果把它写成一个方程, 可以得到:
\begin{center}
\includegraphics[width=.68\textwidth]{../SOURCE/images/softmax-regression-scalarequation.png}
\end{center}
Ⓔ We can "vectorize" this procedure, turning it into a matrix multiplication and vector addition. This is helpful for computational efficiency. (It's also a useful way to think.)\\
我们也可以用向量表示这个计算过程:用矩阵乘法和向量相加.  这有助于提高计算效率(也是一种更有效的思考方式).
\begin{center}
\includegraphics[width=.68\textwidth]{../SOURCE/images/softmax-regression-vectorequation.png}
\end{center}
Ⓔ More compactly, we can just write:\\
更进一步, 可以写成更加紧凑的方式:
\begin{equation}
y = softmax(W_x+b)
\end{equation}

\subsection {实现回归模型}
Ⓔ To do efficient numerical computing in Python, we typically use libraries like NumPy that do expensive operations such as matrix multiplication outside Python, using highly efficient code implemented in another language. Unfortunately, there can still be a lot of overhead from switching back to Python every operation. This overhead is especially bad if you want to run computations on GPUs or in a distributed manner, where there can be a high cost to transferring data.

为了在python中高效的进行数值计算, 我们通常会调用(如NumPy)外部函数库, 把类似矩阵乘法这样的复杂运算使用其他外部语言实现.  不幸的是, 从外部计算切换回Python的每一个操作, 仍然是一个很大的开销.  如果你用GPU来进行外部计算, 这样的开销会更大.  用分布式的计算方式, 也会花费更多的资源用来传输数据.

Ⓔ TensorFlow also does its heavy lifting outside python, but it takes things a step further to avoid this overhead. Instead of running a single expensive operation independently from Python, TensorFlow lets us describe a graph of interacting operations that run entirely outside Python. (Approaches like this can be seen in a few machine learning libraries.)

TensorFlow也把复杂的计算放在python之外完成, 但是为了避免前面说的那些开销, 它做了进一步完善.  TensorFlow不单独地运行单一的复杂计算, 而是让我们可以先用图描述一系列可交互的计算操作, 然后全部一起在Python之外运行.  (这样类似的运行方式, 可以在不少的机器学习库中看到.  )


Ⓔ To use TensorFlow, we need to import it.

使用TensorFlow之前, 首先导入它:

\begin{lstlisting}
import tensorflow as tf
\end{lstlisting}

Ⓔ We describe these interacting operations by manipulating symbolic variables. Let's create one:

我们通过操作符号变量来描述这些可交互的操作单元, 可以用下面的方式创建一个:

\begin{lstlisting}
x = tf.placeholder("float", [None, 784])
\end{lstlisting}\\
Ⓔ \li{x} isn't a specific value. It's a \li{placeholder}, a value that we'll input when we ask TensorFlow to run a computation. We want to be able to input any number of MNIST images, each flattened into a 784-dimensional vector. We represent this as a 2-D tensor of floating-point numbers, with a shape \li{[None, 784]}. (Here None means that a dimension can be of any length.)\\
\li{x} 不是一个特定的值, 而是一个{\em 占位符} \li{placeholder}, 我们在TensorFlow运行计算时输入这个值.  我们希望能够输入任意数量的MNIST图像, 每一张图展平成784维的向量.  我们用2维的浮点数张量来表示这些图, 这个张量的形状是 [None, 784].  (这里的\li{None}表示此张量的第一个维度可以是任何长度的.  )

\index{占位符}
\index{placeholder}

Ⓔ We also need the weights and biases for our model. We could imagine treating these like additional inputs, but TensorFlow has an even better way to handle it: \li{Variable}. A \li{Variable} is a modifiable tensor that lives in TensorFlow's graph of interacting operations. It can be used and even modified by the computation. For machine learning applications, one generally has the model parameters be \li{Variables}.

我们的模型也需要权重值和偏置量, 当然我们可以把它们当做是另外的输入(使用占位符), 但TensorFlow有一个更好的方法来表示它们:\li{Variable}.   \li{Variable}是一个可修改的张量, 存在于 TensorFlow的计算图中, 用于描述交互操作.  它们可以用于计算的输入, 也可以在计算中被修改.  对于各种机器学习应用, 一般都会有模型参数, 用\li{Variable} 变量表示.

\begin{lstlisting}
W = tf.Variable(tf.zeros([784,10]))
b = tf.Variable(tf.zeros([10]))
\end{lstlisting}

Ⓔ We create these \li{Variables} by giving \li{tf.Variable} the initial value of the \li{Variable}: in this case, we initialize both W and b as tensors full of zeros. Since we are going to learn \li{W} and \li{b}, it doesn't matter very much what they initially are.

我们赋予\li{tf.Variable} 不同的初值来创建不同的\li{Variable}:在这里, 我们都用全为零的张量来初始化\li{W}和\li{b}.  因为我们要学习\li{W}和\li{b}的值, 它们的初值可以随意设置.


Ⓔ Notice that \li{W} has a shape of \li{[784, 10]} because we want to multiply the 784-dimensional image vectors by it to produce 10-dimensional vectors of evidence for the difference classes. \li{b} has a shape of \li{[10]} so we can add it to the output.


注意, \li{W} 的维度是 \li{[784, 10]}, 因为我们想要用 784 维的图片向量乘以它以得到一个10维的证据值向量, 每一位对应不同数字类.  \li{b}的形状是\li{[10]}, 所以我们可以直接把它加到输出上面.

Ⓔ We can now implement our model. It only takes one line!

现在, 可以实现我们的模型了, 只需以下一行代码:

\begin{lstlisting}
y = tf.nn.softmax(tf.matmul(x,W) + b)
\end{lstlisting}

Ⓔ First, we multiply $x$ by $W$ with the expression \li{tf.matmul(x, W)}. This is flipped from when we multiplied them in our equation, where we had $W_x$, as a small trick to deal with x being a 2D tensor with multiple inputs. We then add b, and finally apply \li{tf.nn.softmax}.

首先, 我们用 \li{tf.matmul(X, W)} 表示 $x$ 乘以 $W$, 对应之前等式里面的 $W_x$ , 这里 $x$ 是一个 2 维张量拥有多个输入.  然后再加上 $b$, 把和输入到 \li{tf.nn.softmax}函数里面.

Ⓔ That's it. It only took us one line to define our model, after a couple short lines of setup. That isn't because TensorFlow is designed to make a softmax regression particularly easy: it's just a very flexible way to describe many kinds of numerical computations, from machine learning models to physics simulations. And once defined, our model can be run on different devices: your computer's CPU, GPUs, and even phones!

至此, 我们先用了几行简短的代码来设置变量, 然后只用了一行代码来定义我们的模型.  TensorFlow不仅仅可以使softmax回归模型计算变得特别简单, 它也用这种非常灵活的方式来描述其他各种数值计算, 从机器学习模型对物理学模拟仿真模型.  一旦被定义好之后, 我们的模型就可以在不同的设备上运行:计算机的CPU, GPU, 甚至是手机!

\subsection{训练模型}

Ⓔ \cet{In order to train our model, we need to define what it means for the model to be good. Well, actually, in machine learning we typically define what it means for a model to be bad, called the cost or loss, and then try to minimize how bad it is. But the two are equivalent.}

为了训练我们的模型, 我们首先需要定义一个指标来评估这个模型的优劣.  在机器学习中, 我们通常用指标来描述模型劣度, 称之为 {\em 代价} (cost) 或 {\em 损失} (loss), 然后尽量最小化模型的这个指标.  所以这两种称呼方式实质是相同的.

Ⓔ One very common, very nice cost function is "cross-entropy." Surprisingly, cross-entropy arises from thinking about information compressing codes in information theory but it winds up being an important idea in lots of areas, from gambling to machine learning. It's defined:

一个常见的优秀的代价函数就是 {\em 交叉熵} (cross-entropy).  不可思议的是, 交叉熵原本起源于信息论中研究的信息压缩技术, 它后来演变成为从博弈论到机器学习等其他领域里的重要技术手段.  它的定义如下:
\\
\begin{equation}
H_{y'}(y) = -\sum_i{y_{i}'log(y_i)}
\end{equation}
\\
Ⓔ where $y$ is our predicted probability distribution, and $y'$ is the true distribution (the one-hot vector we'll input). In some rough sense, the cross-entropy is measuring how inefficient our predictions are for describing the truth. Going into more detail about cross-entropy is beyond the scope of this tutorial, but it's well worth \href{http://colah.github.io/posts/2015-09-Visual-Information/}{understanding}.
\\
$y$是我们预测的概率分布,$y'$是实际的分布(我们输入的one-hot vector).  比较粗糙的理解是, 交叉熵是用来衡量我们的预测用于描述真相的低效性.  更详细的关于交叉熵的解释超出本教程的范畴, 但是你很有必要好好\href{http://colah.github.io/posts/2015-09-Visual-Information/}{理解}它.

Ⓔ To implement cross-entropy we need to first add a new placeholder to input the correct answers:\\
为了计算交叉熵, 我们首先需要添加一个新的占位符用于输入正确值:
\\
\begin{lstlisting}
y = tf.placeholder("float", [None,10])
\end{lstlisting}
Ⓔ Then we can implement the cross-entropy, $-\sum{y'log(y)}$,\\
然后我们可以用 $-\sum{y'log(y)}$ 计算交叉熵:

\begin{lstlisting}
cross_entropy = tf.reduce_mean(-tf.reduce_sum(y_ * tf.log(y), reduction_indices=[1]))
\end{lstlisting}

Ⓔ First, \li{tf.log} computes the logarithm of each element of \li{y}. Next, we multiply each element of \li{y_} with the corresponding element of \li{tf.log(y)}. Then \li{tf.reduce_sum} adds the elements in the second dimension of \li{y}, due to the \li{reduction_indices = [1]} parameter. Finally, \li{tf.reduce_sum} adds all the elements of the tensor.

首先, 用 \li{tf.log} 计算y的每个元素的对数.  接下来, 我们把\li{y_}的每一个元素和\li{tf.log(y_)}的对应元素相乘.  最后, 用\li{tf.reduce_sum}计算张量的所有元素的总和.

% Ⓔ Note that this isn't just the cross-entropy of the truth with a single prediction, but the sum of the cross-entropies for all the images we looked at. In this example, we have 100 images in each batch: how well we are doing on 100 data points is a much better description of how good our model is than a single data point.

% 值得注意的是, 这里的交叉熵不仅仅用来衡量单一的一对预测和真实值, 而是所有100幅图片的交叉熵的总和.  对于100个数据点的预测表现比单一数据点的表现能更好地描述我们的模型的性能.

Ⓔ Now that we know what we want our model to do, it's very easy to have TensorFlow train it to do so. Because TensorFlow knows the entire graph of your computations, it can automatically use the \href{http://colah.github.io/posts/2015-08-Backprop/}{backpropagation algorithm} to efficiently determine how your variables affect the cost you ask it minimize. Then it can apply your choice of optimization algorithm to modify the variables and reduce the cost.

现在我们知道我们需要我们的模型做什么啦, 用TensorFlow来训练它是非常容易的.  因为TensorFlow拥有一张描述你各个计算单元的图, 它可以自动地使用\href{http://colah.github.io/posts/2015-08-Backprop/}{反向传播算法(backpropagation algorithm)}来有效地确定你的变量是如何影响你想要最小化的那个成本值的.  然后, TensorFlow会用你选择的优化算法来不断地修改变量以降低成本.

\begin{lstlisting}
train_step = tf.train.GradientDescentOptimizer(0.5).minimize(cross_entropy)
\end{lstlisting}

Ⓔ In this case, we ask TensorFlow to minimize \li{cross_entropy} using the gradient descent algorithm with a learning rate of $0.5$. Gradient descent is a simple procedure, where TensorFlow simply shifts each variable a little bit in the direction that reduces the cost. But TensorFlow also provides \href{https://www.tensorflow.org/versions/master/api_docs/python/train.html#optimizers}{many other optimization algorithms}: using one is as simple as tweaking one line.

在这里, 我们要求TensorFlow用梯度下降算法(gradient descent algorithm)以0.01的学习速率最小化交叉熵.  梯度下降算法(gradient descent algorithm)是一个简单的学习过程, TensorFlow只需将每个变量一点点地往使成本不断降低的方向移动.  当然TensorFlow也提供了\href{https://www.tensorflow.org/versions/master/api_docs/python/train.html#optimizers}{其他许多优化算法}:只要简单地调整一行代码就可以使用其他的算法.  \index{梯度下降法}

Ⓔ What TensorFlow actually does here, behind the scenes, is it adds new operations to your graph which implement backpropagation and gradient descent. Then it gives you back a single operation which, when run, will do a step of gradient descent training, slightly tweaking your variables to reduce the cost.

TensorFlow在这里实际上所做的是, 它会在后台给描述你的计算的那张图里面增加一系列新的计算操作单元用于实现反向传播算法和梯度下降算法.  然后, 它返回给你的只是一个单一的操作, 当运行这个操作时, 它用梯度下降算法训练你的模型, 微调你的变量, 不断减少成本.

Ⓔ Now we have our model set up to train. One last thing before we launch it, we have to add an operation to initialize the variables we created:

现在, 我们已经设置好了我们的模型.  在运行计算之前, 我们需要添加一个操作来初始化我们创建的变量:

\begin{lstlisting}
init = tf.initialize_all_variables()
\end{lstlisting}
\\
Ⓔ We can now launch the model in a \li{Session}, and run the operation that initializes the variables:\\
现在我们可以在一个 \li{Session} 里面启动我们的模型, 并且初始化变量:
\begin{lstlisting}
sess = tf.Session()
sess.run(init)
\end{lstlisting}
\\
Ⓔ Let's train -- we'll run the training step 1000 times!\\
然后开始训练模型, 这里我们让模型循环训练1000次!
\begin{lstlisting}
for i in range(1000):
    batch_xs, batch_ys = mnist.train.next_batch(100)
    sess.run(train_step, feed_dict={x: batch_xs, y_: batch_ys})
\end{lstlisting}

Ⓔ Each step of the loop, we get a "batch" of one hundred random data points from our training set. We run \li{train_step} feeding in the batches data to replace the placeholders.

每循环一次, 我们都会随机从训练数据集中抓取``一批'' 100 条数据点, 然后用这些数据点作为参数替换之前的占位符来运行\li{train_step}.

Ⓔ Using small batches of random data is called stochastic training -- in this case, stochastic gradient descent. Ideally, we'd like to use all our data for every step of training because that would give us a better sense of what we should be doing, but that's expensive. So, instead, we use a different subset every time. Doing this is cheap and has much of the same benefit.

每个批次随机使用一小部分训练数据来训练模型被称为随机训练(stochastic training) --- 更确切的说是{\em 随机梯度下降}.  理想情况下, 我们希望用我们所有的数据来进行每一步的训练, 因为这能给我们更好的训练结果, 但显然这需要很大的计算开销.  所以, 每一次训练我们可以使用不同的数据子集, 这样做既可以减少计算开销, 又可以最大化地学习到数据集的总体特性.

\subsection{Evaluating Our Model   ||   评估我们的模型}

Ⓔ How well does our model do?

那么我们的模型性能如何呢?

Ⓔ Well, first let's figure out where we predicted the correct label. \li{tf.argmax} is an extremely useful function which gives you the index of the highest entry in a tensor along some axis. For example, \li{tf.argmax(y,1)} is the label our model thinks is most likely for each input, while \li{tf.argmax(y_,1)} is the correct label. We can use \li{tf.equal} to check if our prediction matches the truth.

首先让我们找出那些预测正确的标签.  \li{tf.argmax()}是一个非常有用的函数, 它能给你在一个张量里沿着某条轴的最高条目的索引值.  比如, \li{tf.argmax(y,1)}是模型认为每个输入最有可能对应的那些标签, 而\li{tf.argmax(y_,1)}代表正确的标签.  我们可以用\li{tf.equal} 来检测我们的预测是否真实标签匹配.

\begin{lstlisting}
correct_prediction = tf.equal(tf.argmax(y,1), tf.argmax(y_,1))
\end{lstlisting}

Ⓔ That gives us a list of booleans. To determine what fraction are correct, we cast to floating point numbers and then take the mean. For example, \li{[True, False, True, True]} would become \li{[1,0,1,1]} which would become $0.75$.

这行代码会给我们一组布尔值.  为了确定正确预测项的比例, 我们可以把布尔值转换成浮点数, 然后取平均值.  例如, \li{[True, False, True, True]}会变成\li{[1,0,1,1]}, 取平均值后得到 $0.75$ .

\begin{lstlisting}
accuracy = tf.reduce_mean(tf.cast(correct_prediction, "float"))
\end{lstlisting}

Ⓔ Finally, we ask for our accuracy on our test data.

最后, 我们计算所学习到的模型在测试数据集上面的正确率.

\begin{lstlisting}
print sess.run(accuracy, feed_dict={x: mnist.test.images, y_: mnist.test.labels})
\end{lstlisting}

Ⓔ This should be about 91\%.

最终结果值应该大约是91\%.

Ⓔ Is that good? Well, not really. In fact, it's pretty bad. This is because we're using a very simple model. With some small changes, we can get to 97\%/. The best models can get to over 99.7\% accuracy! (For more information, have a look at this \href{http://rodrigob.github.io/are_we_there_yet/build/classification_datasets_results.html}{list of results})

这个结果好吗?嗯, 并不太好.  事实上, 这个结果是很差的.  这是因为我们仅仅使用了一个非常简单的模型.  不过, 做一些小小的改进, 我们就可以得到97\%的正确率.  最好的模型甚至可以获得超过99.7\%的准确率!(想了解更多信息, 请参考这个\href{http://rodrigob.github.io/are_we_there_yet/build/classification_datasets_results.html}{结果对比列表}.  )

Ⓔ What matter is that we learned from this model. Still, if you're feeling a bit down about these results, check out the next tutorial where we do a lot better, and learn how to build more sophisticated models using TensorFlow!

比结果更重要的是, 我们从这个模型中学习到的设计思想.  不过, 如果你仍然对这里的结果有点失望, 可以查看下个教程, 在那里你将学到如何用FensorFlow构建更加复杂的模型以获得更好的性能!

原文地址:\href{http://tensorflow.org/tutorials/mnist/beginners/index.md}{MNIST For ML Beginners}

%!TEX program = xelatex
% Encoding: UTF8
% SEIKA 2015


% Chapter 2 Tutorials
% Section 2.3 minist_pros


\newpage
\section {\textcolor{etc}{Deep MNIST for Experts}   ||   深入MNIST} \label{MINIST_pros}

Ⓔ \textcolor{etc}{TensorFlow is a powerful library for doing large-scale numerical computation. One of the tasks at which it excels is implementing and training deep neural networks. In this tutorial we will learn the basic building blocks of a TensorFlow model while constructing a deep convolutional MNIST classifier.}

TensorFlow是一个用于大规模数值计算的强大库件。它的一个强项就是训练并实现深度神经网络(deep neural networks)。在本小节中,我们将会学习TensorFlow模型构建的基本方法,并以此构建一个深度卷积MNIST分类器。

Ⓔ \textcolor{etc}{This introduction assumes familiarity with neural networks and the MNIST dataset. If you don't have a background with them, check out the \hyperref[MINIST_beginner]{introduction for beginners}. Be sure to \hyperref[download_install]{install TensorFlow} before starting.}

本教程假设您已经熟悉神经网络和MNIST数据集。如果你尚未了解,请查看\hyperref[MINIST_beginner]{新手指南}。再开始学习前请确保您已\hyperref[download_install]{安装TensorFlow}。

%
%%
\subsection {Setup  |  安装}

Ⓔ \textcolor{etc}{Before we create our model, we will first load the MNIST dataset, and start a TensorFlow session.}

在创建模型之前,我们会先加载MNIST数据集,然后启动一个TensorFlow会话。

\subsubsection {Load MNIST Data  |  加载MINIST数据}

\textcolor{etc}{For your convenience, we've included \href{https://tensorflow.googlesource.com/tensorflow/+/master/tensorflow/examples/tutorials/mnist/input_data.py}{a script} which automatically downloads and imports the MNIST dataset. It will create a directory \lstinline{'MNIST_data'} in which to store the data files.}

为了方便起见,我们已经准备了一个\href{https://tensorflow.googlesource.com/tensorflow/+/master/tensorflow/examples/tutorials/mnist/input_data.py}{脚本}来自动下载和导入MNIST数据集。它会自动创建一个\lstinline{'MNIST_data'}的目录来存储数据。

\begin{lstlisting}
import input_data
mnist = input_data.read_data_sets('MNIST_data', one_hot=True)
\end{lstlisting}

\textcolor{etc}{Here \lstinline{mnist} is a lightweight class which stores the training, validation, and testing sets as NumPy arrays. It also provides a function for iterating through data minibatches, which we will use below.}

此处的 \lstinline{mnist} 是一个以NumPy数组形式存储训练、验证和测试数据的轻量级类。我们将在之后使用到它提供的一个函数功能,用于迭代按批处理数据。

\subsubsection {Start TensorFlow InteractiveSession  |  开始TensorFlow交互会话}

Ⓔ \textcolor{etc}{Tensorflow relies on a highly efficient C++ backend to do its computation. The connection to this backend is called a session. The common usage for TensorFlow programs is to first create a graph and then launch it in a session.}

Tensorflow基于一个高效的C++后台模块进行运算。与这个后台模块的连接叫做\emph{会话}(session)。TensorFlow编程的常规流程是先创建一个图,然后在session中加载它。

Ⓔ \textcolor{etc}{Here we instead use the convenient InteractiveSession class, which makes TensorFlow more flexible about how you structure your code. It allows you to interleave operations which build a \hyperref[computation_graph]{computation graph} with ones that run the graph. This is particularly convenient when working in interactive contexts like IPython. If you are not using an InteractiveSession, then you should build the entire computation graph before starting a session and \hyperref[launching_graph]{launching} the graph.}

这里,我们使用更加方便的\emph{交互会话}(InteractiveSession)类,它可以让您更加灵活地构建代码。交互会话能让你在运行图的时候,插入一些构建计算图的操作。这能给使用交互式文本shell如iPython带来便利。如果你没有使用InteractiveSession的话,你需要在开始session和加载图之前,构建整个计算图。

\begin{lstlisting}
import tensorflow as tf
sess = tf.InteractiveSession()
\end{lstlisting}

\subsubsection {Computation Graph  |  计算图}

Ⓔ \textcolor{etc}{To do efficient numerical computing in Python, we typically use libraries like NumPy that do expensive operations such as matrix multiplication outside Python, using highly efficient code implemented in another language. Unfortunately, there can still be a lot of overhead from switching back to Python every operation. This overhead is especially bad if you want to run computations on GPUs or in a distributed manner, where there can be a high cost to transferring data.}

为了高效地在Python里进行数值计算,我们一般会使用像NumPy这样用其他语言编写的库件,在Python外用其它执行效率高的语言完成这些高运算开销操作(如矩阵运算)。但是,每一步操作依然会需要切换回Python带来很大开销。特别的,这种开销会在GPU运算或是分布式集群运算这类高数据传输需求的运算形式上非常高昂。

Ⓔ \textcolor{etc}{TensorFlow also does its heavy lifting outside Python, but it takes things a step further to avoid this overhead. Instead of running a single expensive operation independently from Python, TensorFlow lets us describe a graph of interacting operations that run entirely outside Python. This approach is similar to that used in Theano or Torch.}

TensorFlow将高运算量计算放在Python外进行,同时更进一步设法避免上述的额外运算开销。不同于在Python中独立运行运算开销昂贵的操作,TensorFlow让我们可以独立于Python以外以图的形式描述交互式操作。这与Theano、Torch的做法很相似。

Ⓔ \textcolor{etc}{The role of the Python code is therefore to build this external computation graph, and to dictate which parts of the computation graph should be run. See the \hyperref[computation_graph]{Computation Graph} section of \hyperref[basic_usage]{Basic Usage} for more detail.}

因此,这里Python代码的角色是构建其外部将运行的\emph{计算图},并决定计算图的哪一部分将被运行。更多的细节和\hyperref[basic_usage]{基本使用方法}请参阅\hyperref[computation_graph]{计算图}章节。

%
%%
\subsection{Build a Softmax Regression Model  ||  构建 Softmax 回归模型}

Ⓔ \textcolor{etc}{In this section we will build a softmax regression model with a single linear layer. In the next section, we will extend this to the case of softmax regression with a multilayer convolutional network.}

在这小节里,我们将会构建一个包含单个线性隐层的 softmax 回归模型。我们将在下一小结把它扩展成多层卷积网络 softmax回归模型。
\index{Softmax regression}

%
\subsubsection{Placeholder  |  占位符}

Ⓔ \textcolor{etc}{We start building the computation graph by creating nodes for the input images and target output classes.}

我们先从创建输入图像和输出类别的节点来创建计算图。

\begin{lstlisting}
x = tf.placeholder("float", shape=[None, 784])
y_ = tf.placeholder("float", shape=[None, 10])
\end{lstlisting}

Ⓔ \textcolor{etc}{Here \lstinline{x} and \lstinline{y} aren't specific values. Rather, they are each a \lstinline{placeholder} --- a value that we'll input when we ask TensorFlow to run a computation.}

这里的\lstinline{x}和\lstinline{y}并不代表具体值,他们是一个\emph{占位符}(\lstinline{placeholder}) --- 当TensorFlow运行时需要赋值的变量。

Ⓔ \textcolor{etc}{The input images \lstinline{x} will consist of a 2d tensor of floating point numbers. Here we assign it a \lstinline{shape} of \lstinline{[None, 784]}, where \lstinline{784} is the dimensionality of a single flattened MNIST image, and None indicates that the first dimension, corresponding to the batch size, can be of any size. The target output classes \lstinline{y_} will also consist of a 2d tensor, where each row is a one-hot 10-dimensional vector indicating which digit class the corresponding MNIST image belongs to.}

输入图片\lstinline{x}是由浮点数值组成的2维张量(tensor)。这里,我们定义它为\lstinline{[None, 784]}的\lstinline{shape},其中\lstinline{784}是单张展开的MNIST图片的维度数。\lstinline{None}对应\lstinline{shape}的第一个维度,代表了这批输入图像的数量,可能是任意值。目标输出类\lstinline{y_}也是一个2维张量,其中每一行为一个10维向量代表对应MNIST图片的所属数字的类别。

Ⓔ \textcolor{etc}{The shape argument to placeholder is optional, but it allows TensorFlow to automatically catch bugs stemming from inconsistent tensor shapes.}

虽然\lstinline{placeholder}的\lstinline{shape}参数是可选的,但有了它,TensorFlow能够自动捕捉因数据维度不一致导致的错误。


%%%%%
% I, Seika, have revised to here
% \fcolorbox{gray}{yellow}{test 张量 \emph{矢量} 1234 \lstinline{shape}}
%%%%%

\subsubsection{Variables  |  变量}

Ⓔ \textcolor{etc}{We now define the weights \lstinline{W} and biases \lstinline{b} for our model. We could imagine treating these like additional inputs, but TensorFlow has an even better way to handle them: \lstinline{Variable}. A \lstinline{Variable} is a value that lives in TensorFlow's computation graph. It can be used and even modified by the computation. In machine learning applications, one generally has the model parameters be \lstinline{Variables}.}

我们现在为模型定义权重\lstinline{W}和偏置\lstinline{b}。它们可以被视作是额外的输入量,但是TensorFlow有一个更好的方式来处理:\lstinline{Variable}。一个\lstinline{Variable}代表着在TensorFlow计算图中的一个值,它是能在计算过程中被读取和修改的。在机器学习的应用过程中,模型参数一般用\lstinline{Variable}来表示。

\begin{lstlisting}
W = tf.Variable(tf.zeros([784,10]))
b = tf.Variable(tf.zeros([10]))
\end{lstlisting}

Ⓔ \textcolor{etc}{We pass the initial value for each parameter in the call to \lstinline{tf.Variable}. In this case, we initialize both \lstinline{W} and \lstinline{b} as tensors full of zeros. \lstinline{W} is a $784\times10$ matrix (because we have $784$ input features and $10$ outputs) and \lstinline{b} is a 10-dimensional vector (because we have $10$ classes).}

我们在调用\lstinline{tf.Variable}的时候传入初始值。在这个例子里,我们把\lstinline{W}和\lstinline{b}都初始化为零向量。\lstinline{W}是一个$784\times10$的矩阵(因为我们有784个特征和10个输出值)。\lstinline{b}是一个10维的向量(因为我们有10个分类)。

Ⓔ \textcolor{etc}{Before \lstinline{Variables} can be used within a session, they must be initialized using that session. This step takes the initial values (in this case tensors full of zeros) that have already been specified, and assigns them to each \lstinline{Variable}. This can be done for all \lstinline{Variables} at once.}

\lstinline{Variable}需要在\lstinline{session}之前初始化,才能在\lstinline{session}中使用。初始化需要初始值(本例当中是全为零)传入并赋值给每一个\lstinline{Variable}。这个操作可以一次性完成。

\begin{lstlisting}
sess.run(tf.initialize_all_variables())
\end{lstlisting}

%
%%
\subsubsection{Predicted Class and Cost Function  |  预测分类与损失函数}

Ⓔ \textcolor{etc}{We can now implement our regression model. It only takes one line! We multiply the vectorized input images \lstinline{x} by the weight matrix \lstinline{W}, add the bias \lstinline{b}, and compute the softmax probabilities that are assigned to each class.}

现在我们可以实现我们的regression模型了。这只需要一行!我们把图片\lstinline{x}和权重矩阵\lstinline{W}相乘,加上偏置\lstinline{b},然后计算每个分类的softmax概率值。

\begin{lstlisting}
y = tf.nn.softmax(tf.matmul(x,W) + b)
\end{lstlisting}

Ⓔ \textcolor{etc}{The cost function to be minimized during training can be specified just as easily. Our cost function will be the cross-entropy between the target and the model's prediction.}

在训练中最小化损失函数同样很简单。我们这里的损失函数用目标分类和模型预测分类之间的交叉熵。

\begin{lstlisting}
cross_entropy = -tf.reduce_sum(y_*tf.log(y))
\end{lstlisting}

Ⓔ \textcolor{etc}{Note that \lstinline{tf.reduce_sum} sums across all images in the minibatch, as well as all classes. We are computing the cross entropy for the entire minibatch.}

注意,\lstinline{tf.reduce_sum}把\lstinline{minibatch}里的每张图片的交叉熵值都加起来了。我们计算的交叉熵是指整个\lstinline{minibatch}的。

%
%%
\subsection{Train the Model | 训练模型}

Ⓔ \textcolor{etc}{Now that we have defined our model and training cost function, it is straightforward to train using TensorFlow. Because TensorFlow knows the entire computation graph, it can use automatic differentiation to find the gradients of the cost with respect to each of the variables. TensorFlow has a variety of \hyperref[optimizers]{builtin optimization algorithms}. For this example, we will use steepest gradient descent, with a step length of 0.01, to descend the cross entropy.}

我们已经定义好了模型和训练的时候用的损失函数,接下来使用TensorFlow来训练。因为TensorFlow知道整个计算图,它会用自动微分法来找到损失函数对于各个变量的梯度。TensorFlow有大量\hyperref[optimizers]{内置优化算法},这个例子中,我们用快速梯度下降法让交叉熵下降,步长为0.01。

\begin{lstlisting}
train_step = tf.train.GradientDescentOptimizer(0.01).minimize(cross_entropy)
\end{lstlisting}

Ⓔ \textcolor{etc}{What TensorFlow actually did in that single line was to add new operations to the computation graph. These operations included ones to compute gradients, compute parameter update steps, and apply update steps to the parameters.}

这一行代码实际上是用来往计算图上添加一个新操作,其中包括计算梯度,计算每个参数的步长变化,并且计算出新的参数值。

Ⓔ \textcolor{etc}{The returned operation \lstinline{train_step}, when run, will apply the gradient descent updates to the parameters. Training the model can therefore be accomplished by repeatedly running \lstinline{train_step}.}

\lstinline{train_step}这个操作,用梯度下降来更新权值。因此,整个模型的训练可以通过反复地运行\lstinline{train_step}来完成。

\begin{lstlisting}
for i in range(1000):
    batch = mnist.train.next_batch(50)
    train_step.run(feed_dict={x: batch[0], y_: batch[1]})
\end{lstlisting}

Ⓔ \textcolor{etc}{Each training iteration we load 50 training examples. We then run the \lstinline{train_step} operation, using \lstinline{feed_dict} to replace the \lstinline{placeholder} tensors \lstinline{x} and \lstinline{y_} with the training examples. Note that you can replace any tensor in your computation graph using \lstinline{feed_dict} --- it's not restricted to just \lstinline{placeholders}.}

每一步迭代,我们都会加载50个训练样本,然后执行一次\lstinline{train_step,使用}\lstinline{feed_dict},用训练数据替换\lstinline{placeholder}向量\lstinline{x}和\lstinline{y_}。注意,在计算图中,你可以用\lstinline{feed_dict}来替代任何张量,并不仅限于替换\lstinline{placeholder}。

\subsubsection{Evaluate the Model  |  评估模型}

Ⓔ \textcolor{etc}{How well did our model do?}

我们的模型效果怎样?

Ⓔ \textcolor{etc}{First we'll figure out where we predicted the correct label. \lstinline{tf.argmax} is an extremely useful function which gives you the index of the highest entry in a tensor along some axis. For example, \lstinline{tf.argmax(y,1)} is the label our model thinks is most likely for each input, while \lstinline{tf.argmax(y_,1)} is the true label. We can use \lstinline{tf.equal} to check if our prediction matches the truth.}

首先,要先知道我们哪些label是预测正确了。\lstinline{tf.argmax}是一个非常有用的函数。它会返回一个张量某个维度中的最大值的索引。例如,\lstinline{tf.argmax(y,1)}表示我们模型对每个输入的最大概率分类的分类值。而 \lstinline{tf.argmax(y_,1)}表示真实分类值。我们可以用\lstinline{tf.equal}来判断我们的预测是否与真实分类一致。

\begin{lstlisting}
correct_prediction = tf.equal(tf.argmax(y,1), tf.argmax(y_,1))
\end{lstlisting}

Ⓔ \textcolor{etc}{That gives us a list of booleans. To determine what fraction are correct, we cast to floating point numbers and then take the mean. For example, \lstinline{[True, False, True, True]} would become \lstinline{[1,0,1,1]} which would become \lstinline{0.75}.}

这里返回一个布尔数组。为了计算我们分类的准确率,我们将布尔值转换为浮点数来代表对、错,然后取平均值。例如:\lstinline{[True, False, True, True]}变为\lstinline{[1,0,1,1]},计算出平均值为\lstinline{0.75}。

\begin{lstlisting}
accuracy = tf.reduce_mean(tf.cast(correct_prediction, "float"))
\end{lstlisting}

Ⓔ \textcolor{etc}{Finally, we can evaluate our accuracy on the test data. This should be about 91\% correct.}

最后,我们可以计算出在测试数据上的准确率,大概是91\%。

\begin{lstlisting}
print accuracy.eval(feed_dict={x: mnist.test.images, y_: mnist.test.labels})
\end{lstlisting}

%
%%
\subsection{Build a Multilayer Convolutional Network  |  构建多层卷积网络模型}

Ⓔ \textcolor{etc}{Getting 91\% accuracy on MNIST is bad. It's almost embarrassingly bad. In this section, we'll fix that, jumping from a very simple model to something moderately sophisticated: a small convolutional neural network. This will get us to around 99.2\% accuracy --- not state of the art, but respectable.}

在MNIST上只有91\%正确率,实在太糟糕。在这个小节里,我们用一个稍微复杂的模型:卷积神经网络来改善效果。这会达到大概99.2\%的准确率。虽然不是最高,但是还是比较让人满意。
\index{卷积神经网络}

%
\subsubsection{Weight Initialization | 权重初始化}

Ⓔ \textcolor{etc}{To create this model, we're going to need to create a lot of weights and biases. One should generally initialize weights with a small amount of noise for symmetry breaking, and to prevent 0 gradients. Since we're using ReLU neurons, it is also good practice to initialize them with a slightly positive initial bias to avoid "dead neurons." Instead of doing this repeatedly while we build the model, let's create two handy functions to do it for us.}

在创建模型之前,我们先来创建权重和偏置。一般来说,初始化时应加入轻微噪声,来打破对称性,防止零梯度的问题。因为我们用的是ReLU,所以用稍大于0的值来初始化偏置能够避免节点输出恒为0的问题(dead neurons)。为了不在建立模型的时候反复做初始化操作,我们定义两个函数用于初始化。

\begin{lstlisting}
def weight_variable(shape):
    initial = tf.truncated_normal(shape, stddev=0.1)
    return tf.Variable(initial)

def bias_variable(shape):
    initial = tf.constant(0.1, shape=shape)
    return tf.Variable(initial)
\end{lstlisting}

\subsubsection{Convolution and Pooling  |  卷积和池化}

Ⓔ \textcolor{etc}{TensorFlow also gives us a lot of flexibility in convolution and pooling operations. How do we handle the boundaries? What is our stride size? In this example, we're always going to choose the vanilla version. Our convolutions uses a stride of one and are zero padded so that the output is the same size as the input. Our pooling is plain old max pooling over $2\times2$ blocks. To keep our code cleaner, let's also abstract those operations into functions.}

TensorFlow在卷积和池化上有很强的灵活性。我们怎么处理边界?步长应该设多大?在这个实例里,我们会一直使用vanilla版本。我们的卷积使用1步长(stride size),0边距(padding size)的模板,保证输出和输入是同一个大小。我们的池化用简单传统的$2\times2$大小的模板做max pooling。为了代码更简洁,我们把这部分抽象成一个函数。

\begin{lstlisting}
def conv2d(x, W):
    return tf.nn.conv2d(x, W, strides=[1, 1, 1, 1], padding='SAME')

def max_pool_2x2(x):
    return tf.nn.max_pool(x, ksize=[1, 2, 2, 1], strides=[1, 2, 2, 1], padding='SAME')
\end{lstlisting}

\subsubsection{First Convolutional Layer  |  第一层卷积}

Ⓔ \textcolor{etc}{We can now implement our first layer. It will consist of convolution, followed by max pooling. The convolutional will compute 32 features for each $5\times5$ patch. Its weight tensor will have a shape of \lstinline{[5, 5, 1, 32]}. The first two dimensions are the patch size, the next is the number of input channels, and the last is the number of output channels. We will also have a bias vector with a component for each output channel.}

现在我们可以开始实现第一层了。它由一个卷积接一个max pooling完成。卷积在每个$5\times5$的patch中算出32个特征。权重是一个\lstinline{[5, 5, 1, 32]}的张量,前两个维度是patch的大小,接着是输入的通道数目,最后是输出的通道数目。输出对应一个同样大小的偏置向量。

\begin{lstlisting}
W_conv1 = weight_variable([5, 5, 1, 32])
b_conv1 = bias_variable([32])
\end{lstlisting}

Ⓔ \textcolor{etc}{To apply the layer, we first reshape \lstinline{x} to a 4d tensor, with the second and third dimensions corresponding to image width and height, and the final dimension corresponding to the number of color channels.}

为了用这一层,我们把\lstinline{x}变成一个4d向量,第2、3维对应图片的宽高,最后一维代表颜色通道。

\begin{lstlisting}
x_image = tf.reshape(x, [-1,28,28,1])
\end{lstlisting}

Ⓔ \textcolor{etc}{We then convolve \lstinline{x_image} with the weight tensor, add the bias, apply the ReLU function, and finally max pool.}

我们把\lstinline{x_image}和权值向量进行卷积相乘,加上偏置,使用ReLU激活函数,最后max pooling。

\begin{lstlisting}
h_conv1 = tf.nn.relu(conv2d(x_image, W_conv1) + b_conv1)
h_pool1 = max_pool_2x2(h_conv1)
\end{lstlisting}

%
\subsubsection{Second Convolutional Layer  |  第二层卷积}

Ⓔ \textcolor{etc}{In order to build a deep network, we stack several layers of this type. The second layer will have 64 features for each $5\times5$ patch.}

为了构建一个更深的网络,我们会把几个类似的层堆叠起来。第二层中,每个5x5的patch会得到64个特征。

\begin{lstlisting}
W_conv2 = weight_variable([5, 5, 32, 64])
b_conv2 = bias_variable([64])

h_conv2 = tf.nn.relu(conv2d(h_pool1, W_conv2) + b_conv2)
h_pool2 = max_pool_2x2(h_conv2)
\end{lstlisting}

%
\subsubsection{Densely Connected Layer  |  密集连接层}

Ⓔ \textcolor{etc}{Now that the image size has been reduced to $7\times7$, we add a fully-connected layer with 1024 neurons to allow processing on the entire image. We reshape the tensor from the pooling layer into a batch of vectors, multiply by a weight matrix, add a bias, and apply a ReLU.}

现在,图片降维到$7\times7$,我们加入一个有1024个神经元的全连接层,用于处理整个图片。我们把池化层输出的张量reshape成一些向量,乘上权重矩阵,加上偏置,使用ReLU激活。

\begin{lstlisting}
W_fc1 = weight_variable([7 * 7 * 64, 1024])
b_fc1 = bias_variable([1024])

h_pool2_flat = tf.reshape(h_pool2, [-1, 7*7*64])
h_fc1 = tf.nn.relu(tf.matmul(h_pool2_flat, W_fc1) + b_fc1)
\end{lstlisting}

\textbf{Dropout}

Ⓔ \textcolor{etc}{To reduce overfitting, we will apply dropout before the readout layer. We create a placeholder for the probability that a neuron's output is kept during dropout. This allows us to turn dropout on during training, and turn it off during testing. TensorFlow's tf.nn.dropout op automatically handles scaling neuron outputs in addition to masking them, so dropout just works without any additional scaling.}

为了减少过拟合,我们在输出层之前加入dropout。我们用一个placeholder来代表一个神经元在dropout中被保留的概率。这样我们可以在训练过程中启用dropout,在测试过程中关闭dropout。 TensorFlow的\lstinline{tf.nn.dropout}操作会自动处理神经元输出值的scale。所以用dropout的时候可以不用考虑scale。

\begin{lstlisting}
keep_prob = tf.placeholder("float")
h_fc1_drop = tf.nn.dropout(h_fc1, keep_prob)
\end{lstlisting}

%
\subsubsection{Readout Layer  |  输出层}

Ⓔ \textcolor{etc}{Finally, we add a softmax layer, just like for the one layer softmax regression above.}

最后,我们添加一个softmax层,就像前面的单层softmax regression一样。

\begin{lstlisting}
W_fc2 = weight_variable([1024, 10])
b_fc2 = bias_variable([10])

y_conv=tf.nn.softmax(tf.matmul(h_fc1_drop, W_fc2) + b_fc2)
\end{lstlisting}

%
\subsubsection{Train and Evaluate the Model  |  训练和评估模型}

Ⓔ \textcolor{etc}{How well does this model do? To train and evaluate it we will use code that is nearly identical to that for the simple one layer SoftMax network above. The differences are that: we will replace the steepest gradient descent optimizer with the more sophisticated ADAM optimizer; we will include the additional parameter \lstinline{keep_prob} in \lstinline{feed_dict} to control the dropout rate; and we will add logging to every 100th iteration in the training process.}

这次效果又有多好呢?我们用前面几乎一样的代码来测测看。只是我们会用更加复杂的ADAM优化器来做梯度最速下降,在feed\_dict中加入额外的参数keep\_prob来控制dropout比例。然后每100次迭代输出一次日志。

\begin{lstlisting}
cross_entropy = -tf.reduce_sum(y_*tf.log(y_conv))
train_step = tf.train.AdamOptimizer(1e-4).minimize(cross_entropy)
correct_prediction = tf.equal(tf.argmax(y_conv,1), tf.argmax(y_,1))
accuracy = tf.reduce_mean(tf.cast(correct_prediction, "float"))
sess.run(tf.initialize_all_variables())
for i in range(20000):
  batch = mnist.train.next_batch(50)
  if i%100 == 0:
    train_accuracy = accuracy.eval(feed_dict={
        x:batch[0], y_: batch[1], keep_prob: 1.0})
    print "step %d, training accuracy %g"%(i, train_accuracy)
  train_step.run(feed_dict={x: batch[0], y_: batch[1], keep_prob: 0.5})

print "test accuracy %g"%accuracy.eval(feed_dict={
    x: mnist.test.images, y_: mnist.test.labels, keep_prob: 1.0})
\end{lstlisting}

Ⓔ \textcolor{etc}{The final test set accuracy after running this code should be approximately 99.2\%.}

以上代码,在最终测试集上的准确率大概是99.2\%。

Ⓔ \textcolor{etc}{We have learned how to quickly and easily build, train, and evaluate a fairly sophisticated deep learning model using TensorFlow.}

目前为止,我们已经学会了用TensorFlow来快速和简易地搭建、训练和评估一个复杂一点儿的深度学习模型。

原文地址:\href{https://www.tensorflow.org/versions/master/tutorials/mnist/pros/index.html#deep-mnist-for-experts}{Deep MNIST for Experts}

%!TEX program = xelatex
% Encoding: UTF8
% SEIKA 2015


% Chapter 2 Tutorials
% Section 2.4 TensorFlow Mechanics 101


\newpage
\section {TensorFlow Mechanics 101} \label{tf_mech101}

Ⓔ \textcolor{etc}{\textbf{Code}: \href{https://tensorflow/g3doc/tutorials/mnist/}{tensorflow/examples/tutorials/mnist/}}

\textbf{代码地址}: \href{https://tensorflow/g3doc/tutorials/mnist/}{tensorflow/g3doc/tutorials/mnist/}

Ⓔ \textcolor{etc}{The goal of this tutorial is to show how to use TensorFlow to train and evaluate a simple feed-forward neural network for handwritten digit classification using the (classic) MNIST data set. The intended audience for this tutorial is experienced machine learning users interested in using TensorFlow.}

本篇教程的目的,是向大家展示如何利用TensorFlow使用(经典)MNIST数据集训练并评估一个用于识别手写数字的简易前馈神经网络(feed-forward neural network)。我们的目标读者是有兴趣使用TensorFlow的机器学习资深人士。

Ⓔ \textcolor{etc}{These tutorials are not intended for teaching Machine Learning in general.}

因此,撰写该系列教程并不是为了教大家机器学习领域的基础知识。

Ⓔ \textcolor{etc}{Please ensure you have followed the instructions to \href{https://www.tensorflow.org/versions/master/get_started/os_setup.html}{install TensorFlow}.}

在学习本教程之前,请确保您已按照安装TensorFlow教程中的要求,完成了\href{https://www.tensorflow.org/versions/master/get_started/os_setup.html}{安装}。

\subsection {教程使用的文件} \label{minist_tf}

本教程引用如下文件:

% add table here

只需要直接运行\lstinline{fully_connected_feed.py}文件,就可以开始训练:

\begin{lstlisting}
python fully\_connected\_feed.py
\end{lstlisting}


\subsection {准备数据}

Ⓔ \textcolor{etc}{MNIST is a classic problem in machine learning. The problem is to look at greyscale $28 \times 28$ pixel images of handwritten digits and determine which digit the image represents, for all the digits from zero to nine.}

MNIST是机器学习领域的一个经典问题,指的是让机器查看一系列大小为$28\times28$像素的手写数字灰度图像,并判断这些图像代表0--9中的哪一个数字。

\begin{figure}[htbp]
\centering
\includegraphics[width=.55\textwidth]{../SOURCE/images/MNIST.png}
\caption{}
\end{figure}

Ⓔ \textcolor{etc}{For more information, refer to \href{http://yann.lecun.com/exdb/mnist/}{Yann LeCun's MNIST page} or \href{http://colah.github.io/posts/2014-10-Visualizing-MNIST/}{Chris Olah's visualizations of MNIST}.}

更多相关信息,请查阅Yann LeCun网站中关于MNIST的介绍 或者Chris Olah对MNIST的可视化探索。

\subsubsection{Download | 下载}

Ⓔ \textcolor{etc}{At the top of the \lstinline{run_training()} method, the \lstinline{input_data.read_data_sets()} function will ensure that the correct data has been downloaded to your local training folder and then unpack that data to return a dictionary of \lstinline{DataSet} instances.}

在\lstinline{run_training()}方法的一开始,\lstinline{input_data.read_data_sets()}函数会确保你的本地训练文件夹中,已经下载了正确的数据,然后将这些数据解压并返回一个含有\lstinline{DataSet}实例的字典。

\begin{lstlisting}
data_sets = input_data.read_data_sets(FLAGS.train_dir, FLAGS.fake_data)
\end{lstlisting}\footnote{\lstinline{fake_data}标记是用于单元测试的,读者可以不必理会。}

% 数据集 | 目的
% --- | ---
% `data_sets.train` | 55000个图像和标签(labels),作为主要训练集。
% `data_sets.validation` | 5000个图像和标签,用于迭代验证训练准确度。
% `data_sets.test` | 10000个图像和标签,用于最终测试训练准确度(trained accuracy)。

Add table here\footnote{了解更多数据有关信息,请查阅此系列教程的\href{https://www.tensorflow.org/versions/master/tutorials/mnist/download/index.html}{数据下载}部分.}

\subsubsection{Inputs and Placeholders | 输入与占位符}

Ⓔ \textcolor{etc}{The \lstinline{placeholder_inputs()} function creates two \lstinline{tf.placeholder} ops that define the shape of the inputs, including the \lstinline{batch_size}, to the rest of the graph and into which the actual training examples will be fed.}

\lstinline{placeholder_inputs()}函数将生成两个\lstinline{tf.placeholder}
%[`tf.placeholder`](../api_docs/python/io_ops.md#placeholder)
操作,定义传入图表中的shape参数,shape参数中包括\lstinline{batch_size}值,后续还会将实际的训练用例传入图表。

\begin{lstlisting}
images_placeholder = tf.placeholder(tf.float32, shape=(batch_size, IMAGE_PIXELS))
labels_placeholder = tf.placeholder(tf.int32, shape=(batch_size))
\end{lstlisting}

在训练循环(training loop)的后续步骤中,传入的整个图像和标签数据集会被切片,以符合每一个操作所设置的 \lstinline{batch_size}值,占位符操作将会填补以符合这个\lstinline{batch_size}值。然后使用\lstinline{feed_dict}参数,将数据传入\lstinline{sess.run()}函数。

\subsection {Build the Graph | 构建图表}

在为数据创建占位符之后,就可以运行\lstinline{mnist.py}文件,经过三阶段的模式函数操作:\lstinline{inference()}, \lstinline{loss()},和\lstinline{training()}。图表就构建完成了。

\begin{enumerate}

\item \lstinline{inference()} —— 尽可能地构建好图表,满足促使神经网络向前反馈并做出预测的要求。

\item \lstinline{loss()} —— 往inference图表中添加生成损失(loss)所需要的操作(ops)。

\item \lstinline{training()} —— 往损失图表中添加计算并应用梯度(gradients)所需的操作。
\end{enumerate}

\begin{figure}[htbp]
\centering
\includegraphics[width=.95\textwidth]{../SOURCE/images/mnist_subgraph.png}
\caption{}
\end{figure}

\subsubsection{推理(Inference)}
%### 推理(Inference) <a class="md-anchor" id="AUTOGENERATED-inference"></a>

\lstinline{inference()}函数会尽可能地构建图表,做到返回包含了预测结果(output prediction)的Tensor。

它接受图像占位符为输入,在此基础上借助ReLu(Rectified Linear Units)激活函数,构建一对完全连接层(layers),以及一个有着十个节点(node)、指明了输出logtis模型的线性层。

每一层都创建于一个唯一的\lstinline{tf.name_scope}%(../api_docs/python/framework.md#name_scope)
之下,创建于该作用域之下的所有元素都将带有其前缀。

\begin{lstlisting}
with tf.name_scope('hidden1') as scope:
\end{lstlisting}

在定义的作用域中,每一层所使用的权重和偏差都在\lstinline{tf.Variable}
%(../api_docs/python/state_ops.md#Variable)
实例中生成,并且包含了各自期望的shape。

\begin{lstlisting}
weights = tf.Variable(tf.truncated_normal([IMAGE_PIXELS, hidden1_units], stddev=1.0 / math.sqrt(float(IMAGE_PIXELS))), name='weights')
biases = tf.Variable(tf.zeros([hidden1_units]), name='biases')
\end{lstlisting}

例如,当这些层是在\lstinline{hidden1}作用域下生成时,赋予权重变量的独特名称将会是"\lstinline{hidden1/weights}"。

每个变量在构建时,都会获得初始化操作(initializer ops)。

在这种最常见的情况下,通过\lstinline{tf.truncated_normal}
%(../api_docs/python/constant_op.md#truncated_normal)
函数初始化权重变量,给赋予的shape则是一个二维tensor,其中第一个维度代表该层中权重变量所连接(connect from)的单元数量,第二个维度代表该层中权重变量所连接到的(connect to)单元数量。对于名叫\lstinline{hidden1}的第一层,相应的维度则是\lstinline{[IMAGE_PIXELS, hidden1_units]},因为权重变量将图像输入连接到了\lstinline{hidden1}层。\lstinline{tf.truncated_normal}初始函数将根据所得到的均值和标准差,生成一个随机分布。

然后,通过\lstinline{tf.zeros}
%(../api_docs/python/constant_op.md#zeros)
函数初始化偏差变量(biases),确保所有偏差的起始值都是0,而它们的shape则是其在该层中所接到的(connect to)单元数量。

图表的三个主要操作,分别是两个\lstinline{tf.nn.relu}
%(../api_docs/python/nn.md#relu)
操作,它们中嵌入了隐藏层所需的\lstinline{tf.matmul}
%(../api_docs/python/math_ops.md#matmul)
;以及logits模型所需的另外一个\lstinline{tf.matmul}。三者依次生成,各自的\lstinline{tf.Variable}实例则与输入占位符或下一层的输出tensor所连接。

\begin{lstlisting}
hidden1 = tf.nn.relu(tf.matmul(images, weights) + biases)
\end{lstlisting}

\begin{lstlisting}
hidden2 = tf.nn.relu(tf.matmul(hidden1, weights) + biases)
\end{lstlisting}

\begin{lstlisting}
logits = tf.matmul(hidden2, weights) + biases
\end{lstlisting}

最后,程序会返回包含了输出结果的`logits`Tensor。

\subsubsection{损失(Loss)}

\lstinline{loss()}函数通过添加所需的损失操作,进一步构建图表。

首先,\lstinline{labels_placeholer}中的值,将被编码为一个含有1-hot values的Tensor。例如,如果类标识符为“3”,那么该值就会被转换为:
\lstinline{[0, 0, 0, 1, 0, 0, 0, 0, 0, 0]}

\begin{lstlisting}
batch_size = tf.size(labels)
labels = tf.expand_dims(labels, 1)
indices = tf.expand_dims(tf.range(0, batch_size, 1), 1)
concated = tf.concat(1, [indices, labels])
onehot_labels = tf.sparse_to_dense(
    concated, tf.pack([batch_size, NUM_CLASSES]), 1.0, 0.0)
\end{lstlisting}

之后,又添加一个\lstinline{tf.nn.softmax_cross_entropy_with_logits}
%`](../api_docs/python/nn.md#softmax_cross_entropy_with_logits)
操作\footnote{交叉熵是信息理论中的概念,可以让我们描述如果基于已有事实,相信神经网络所做的推测最坏会导致什么结果。更多详情,请查阅博文《可视化信息理论》(http://colah.github.io/posts/2015-09-Visual-Information/)},用来比较\lstinline{inference()}函数与1-hot标签所输出的logits Tensor。

\begin{lstlisting}
cross_entropy = tf.nn.softmax_cross_entropy_with_logits(logits, onehot_labels, name='xentropy')
\end{lstlisting}

然后,使用\lstinline{tf.reduce_mean}
%(../api_docs/python/math_ops.md#reduce_mean)
函数,计算batch维度(第一维度)下交叉熵(cross entropy)的平均值,将将该值作为总损失。

\begin{lstlisting}
loss = tf.reduce_mean(cross_entropy, name='xentropy_mean')
\end{lstlisting}

最后,程序会返回包含了损失值的Tensor。

\subsubsection{训练}

\lstinline{training()}函数添加了通过梯度下降(gradient descent)将损失最小化所需的操作。

首先,该函数从\lstinline{loss()}函数中获取损失Tensor,将其交给\lstinline{[tf.scalar_summary]}
% ](../api_docs/python/train.md#scalar_summary)
,后者在与\lstinline{SummaryWriter}(见下文)配合使用时,可以向事件文件(events file)中生成汇总值(summary values)。在本篇教程中,每次写入汇总值时,它都会释放损失Tensor的当前值(snapshot value)。

\begin{lstlisting}
tf.scalar_summary(loss.op.name, loss)
\end{lstlisting}

接下来,我们实例化一个\lstinline{[tf.train.GradientDescentOptimizer]}
% (../api_docs/python/train.md#GradientDescentOptimizer)
,负责按照所要求的学习效率(learning rate)应用梯度下降法(gradients)。

\begin{lstlisting}
optimizer = tf.train.GradientDescentOptimizer(FLAGS.learning_rate)
\end{lstlisting}

之后,我们生成一个变量用于保存全局训练步骤(global training step)的数值,并使用\lstinline{minimize()}
% (../api_docs/python/train.md#Optimizer.minimize)
函数更新系统中的三角权重(triangle weights)、增加全局步骤的操作。根据惯例,这个操作被称为\lstinline{train_op},是TensorFlow会话(session)诱发一个完整训练步骤所必须运行的操作(见下文)。

\begin{lstlisting}
global_step = tf.Variable(0, name='global_step', trainable=False)
train_op = optimizer.minimize(loss, global_step=global_step)
\end{lstlisting}

最后,程序返回包含了训练操作(training op)输出结果的Tensor。

\subsection{训练模型}

一旦图表构建完毕,就通过\lstinline{fully_connected_feed.py}文件中的用户代码进行循环地迭代式训练和评估。

\subsubsection{图表 (The Graph)}

在\lstinline{run_training()}这个函数的一开始,是一个Python语言中的\lstinline{with}命令,这个命令表明所有已经构建的操作都要与默认的\lstinline{[`tf.Graph`]}
%(../api_docs/python/framework.md#Graph)
全局实例关联起来。

\begin{lstlisting}
with tf.Graph().as_default():
\end{lstlisting}

\lstinline{tf.Graph}实例是一系列可以作为整体执行的操作。TensorFlow的大部分场景只需要依赖默认图表一个实例即可。

利用多个图表的更加复杂的使用场景也是可能的,但是超出了本教程的范围。

\subsubsection{会话 (The Session)}

完成全部的构建准备、生成全部所需的操作之后,我们就可以创建一个\lstinline{tf.Session}
%(../api_docs/python/client.md#Session)
,用于运行图表。

\begin{lstlisting}
sess = tf.Session()
\end{lstlisting}

另外,也可以利用\lstinline{with}代码块生成\lstinline{Session},限制作用域:

\begin{lstlisting}
with tf.Session() as sess:
\end{lstlisting}

\lstinline{Session}函数中没有传入参数,表明该代码将会依附于(如果还没有创建会话,则会创建新的会话)默认的本地会话。

生成会话之后,所有\lstinline{tf.Variable}实例都会立即通过调用各自初始化操作中的\lstinline{sess.run()}
%(../api_docs/python/client.md#Session.run)
函数进行初始化。

\begin{lstlisting}
init = tf.initialize_all_variables()
sess.run(init)
\end{lstlisting}

\lstinline{sess.run()}
%(../api_docs/python/client.md#Session.run)
方法将会运行图表中与作为参数传入的操作相对应的完整子集。在初次调用时,\lstinline{init}操作只包含了变量初始化程序\lstinline{tf.group}
%(../api_docs/python/control_flow_ops.md#group)
。图表的其他部分不会在这里,而是在下面的训练循环运行。

\subsubsection{训练循环}

完成会话中变量的初始化之后,就可以开始训练了。

训练的每一步都是通过用户代码控制,而能实现有效训练的最简单循环就是:

\begin{lstlisting}
for step in xrange(max_steps):
    sess.run(train_op)
\end{lstlisting}

但是,本教程中的例子要更为复杂一点,原因是我们必须把输入的数据根据每一步的情况进行切分,以匹配之前生成的占位符。

\paragragh{向图表提}

执行每一步时,我们的代码会生成一个反馈字典(feed dictionary),其中包含对应步骤中训练所要使用的例子,这些例子的哈希键就是其所代表的占位符操作。

\lstinline{fill_feed_dict}函数会查询给定的\lstinline{DataSet},索要下一批次\lstinline{batch_size}的图像和标签,与占位符相匹配的Tensor则会包含下一批次的图像和标签。

\begin{lstlisting}
images_feed, labels_feed = data_set.next_batch(FLAGS.batch_size)
\end{lstlisting}

然后,以占位符为哈希键,创建一个Python字典对象,键值则是其代表的反馈Tensor。

\begin{lstlisting}
feed_dict = {
    images_placeholder: images_feed,
    labels_placeholder: labels_feed,
}
\end{lstlisting}

这个字典随后作为\lstinline{feed_dict}参数,传入\lstinline{sess.run()}函数中,为这一步的训练提供输入样例。

\paragragh{检查状态}

在运行\lstinline{sess.run()}函数时,要在代码中明确其需要获取的两个值:\lstinline{[train_op, loss]}。

\begin{lstlisting}
for step in xrange(FLAGS.max_steps):
    feed_dict = fill_feed_dict(data_sets.train, images_placeholder, labels_placeholder)
    _, loss_value = sess.run([train_op, loss], feed_dict=feed_dict)
\end{lstlisting}

因为要获取这两个值,\lstinline{sess.run()}会返回一个有两个元素的元组。其中每一个\lstinline{Tensor}对象,对应了返回的元组中的numpy数组,而这些数组中包含了当前这步训练中对应Tensor的值。由于\lstinline{train_op}并不会产生输出,其在返回的元祖中的对应元素就是\lstinline{None},所以会被抛弃。但是,如果模型在训练中出现偏差,\lstinline{loss} Tensor的值可能会变成\lstinline{NaN},所以我们要获取它的值,并记录下来。

假设训练一切正常,没有出现\lstinline{NaN},训练循环会每隔100个训练步骤,就打印一行简单的状态文本,告知用户当前的训练状态。

\begin{lstlisting}
if step % 100 == 0:
    print 'Step %d: loss = %.2f (%.3f sec)' % (step, loss_value, duration)
\end{lstlisting}

\paragraph{状态可视化}
为了释放\lstinline{[TensorBoard]}
%(../how_tos/summaries_and_tensorboard.md)
所使用的事件文件(events file),所有的即时数据(在这里只有一个)都要在图表构建阶段合并至一个操作(op)中。

\begin{lstlisting}
summary_op = tf.merge_all_summaries()
\end{lstlisting}

在创建好会话(session)之后,可以实例化一个\lstinline{tf.train.SummaryWriter}
% (../api_docs/python/train.md#SummaryWriter)
,用于写入包含了图表本身和即时数据具体值的事件文件。

\begin{lstlisting}
summary_writer = tf.train.SummaryWriter(FLAGS.train_dir, graph_def=sess.graph_def)
\end{lstlisting}

最后,每次运行\lstinline{summary_op}时,都会往事件文件中写入最新的即时数据,函数的输出会传入事件文件读写器(writer)的\lstinline{add_summary()}函数。。

\begin{lstlisting}
summary_str = sess.run(summary_op, feed_dict=feed_dict)
summary_writer.add_summary(summary_str, step)
\end{lstlisting}

事件文件写入完毕之后,可以就训练文件夹打开一个TensorBoard,查看即时数据的情况。

\begin{figure}[htbp]
\centering
\includegraphics[width=.85\textwidth]{../SOURCE/images/mnist_tensorboard.png}
\centering
\end{figure}

% ![MNIST TensorBoard](../images/mnist_tensorboard.png "MNIST TensorBoard")

**注意**:了解更多如何构建并运行TensorBoard的信息,请查看相关教程\hyperref[vis_learning]{Tensorboard:训练过程可视化}。

\paragraph{保存检查点(checkpoint)}

为了得到可以用来后续恢复模型以进一步训练或评估的检查点文件(checkpoint file),我们实例化一个\lstinline{tf.train.Saver}
% (../api_docs/python/state_ops.md#Saver)。

\begin{lstlisting}
saver = tf.train.Saver()
\end{lstlisting}

在训练循环中,将定期调用\lstinline{saver.save()}
%(../api_docs/python/state_ops.md#Saver.save)
方法,向训练文件夹中写入包含了当前所有可训练变量值得检查点文件。

\begin{lstlisting}
saver.save(sess, FLAGS.train_dir, global_step=step)
\end{lstlisting}

这样,我们以后就可以使用\lstinline{saver.restore()}
%(../api_docs/python/state_ops.md#Saver.restore)
方法,重载模型的参数,继续训练。

\begin{lstlisting}
saver.restore(sess, FLAGS.train_dir)
\end{lstlisting}

\subsection{评估模型}

每隔一千个训练步骤,我们的代码会尝试使用训练数据集与测试数据集,对模型进行评估。\lstinline{do_eval}函数会被调用三次,分别使用训练数据集、验证数据集合测试数据集。

\begin{lstlisting}
print 'Training Data Eval:'
do_eval(sess, eval_correct, images_placeholder, labels_placeholder, data_sets.train)
print 'Validation Data Eval:'
do_eval(sess, eval_correct, images_placeholder, labels_placeholder, data_sets.validation)
print 'Test Data Eval:'
do_eval(sess, eval_correct, images_placeholder, labels_placeholder, data_sets.test)
\end{lstlisting}

>注意,更复杂的使用场景通常是,先隔绝\lstinline{data_sets.test}测试数据集,只有在大量的超参数优化调整(hyperparameter tuning)之后才进行检查。但是,由于MNIST问题比较简单,我们在这里一次性评估所有的数据。

\subsubsection {构建评估图表(Eval Graph)}

在打开默认图表(Graph)之前,我们应该先调用\lstinline{get_data(train=False)}函数,抓取测试数据集。

\begin{lstlisting}
test_all_images, test_all_labels = get_data(train=False)
\end{lstlisting}

在进入训练循环之前,我们应该先调用\lstinline{mnist.py}文件中的\lstinline{evaluation}函数,传入的logits和标签参数要与\lstinline{loss}函数的一致。这样做事为了先构建Eval操作。

\begin{lstlisting}
eval_correct = mnist.evaluation(logits, labels_placeholder)
\end{lstlisting}

\lstinline{evaluation}函数会生成\hyperref[(../api_docs/python/nn.md#in_top_k)]{\lstinline{tf.nn.in_top_k}}
操作,如果在$k$个最有可能的预测中可以发现真的标签,那么这个操作就会将模型输出标记为正确。在本文中,我们把$k$的值设置为1,也就是只有在预测是真的标签时,才判定它是正确的。

\begin{lstlisting}
eval_correct = tf.nn.in_top_k(logits, labels, 1)
\end{lstlisting}

\subsubsection {评估图表的输出(Eval Output)}

之后,我们可以创建一个循环,往其中添加\lstinline{feed_dict},并在调用\lstinline{sess.run()}函数时传入\lstinline{eval_correct}操作,目的就是用给定的数据集评估模型。

\begin{lstlisting}
for step in xrange(steps_per_epoch):
    feed_dict = fill_feed_dict(data_set,
                               images_placeholder,
                               labels_placeholder)
    true_count += sess.run(eval_correct, feed_dict=feed_dict)
\end{lstlisting}

\lstinline{true_count}变量会累加所有\lstinline{in_top_k}操作判定为正确的预测之和。接下来,只需要将正确测试的总数,除以例子总数,就可以得出准确率了。

\begin{lstlisting}
precision = float(true_count) / float(num_examples)
print '  Num examples: %d  Num correct: %d  Precision @ 1: %0.02f' % (
    num_examples, true_count, precision)
\end{lstlisting}

原文:\href{http://www.tensorflow.org/tutorials/mnist/tf/index.md}{TensorFlow Mechanics 101}
翻译:\href{https://github.com/bingjin}{bingjin}
校对:\href{https://github.com/LichAmnesia}{LichAmnesia}
\include{tutorials/c2s04_}
\include{tutorials/c2s05_word2vec}
\include{tutorials/c2s06_recurrent}
\include{tutorials/c2s07_seq2seq}
\include{tutorials/c2s08_mandelbrot}
\include{tutorials/c2s09_pdes}
\include{tutorials/c2s10_mnist_download}

\newpage
% Chapter 3 How to...
% 第三章 运作方式
\chapter{运作方式}
\include{how_tos/c3s00_overview}
%!TEX program = xelatex
% Encoding: UTF8
% SEIKA 2015


% Chapter 3 How to ...
% Section 3.1



\section{变量:创建、初始化、保存和加载}
\label{variables}

当训练模型时, 用变量来存储和更新参数。  变量包含张量 (Tensor)存放于内存的缓存区。  建模时它们需要被明确地初始化, 模型训练后它们必须被存储到磁盘。  这些变量的值可在之后模型训练和分析是被加载。

本文档描述以下两个TensorFlow类。  点击以下链接可查看完整的API文档:
\begin{itemize}
  \item \li{tf.Variable} 类 % add link here
  \item \li{tf.train.Saver} 类 % add link here
\end{itemize}

\subsection {变量创建}

当创建一个变量时, 你将一个张量作为初始值传入构造函数Variable()。  TensorFlow提供了一系列操作符来初始化张量, 初始值是常量或是随机值。
% add link here

\begin{lstlisting}
# Create two variables.
weights = tf.Variable(tf.random_normal([784, 200], stddev=0.35), name="weights")
biases = tf.Variable(tf.zeros([200]), name="biases")
\end{lstlisting}

调用 \li{tf.Variable()) 添加一些操作(Op, operation)到graph:
\begin{itemize}
  \item 一个Variable操作存放变量的值。
  \item 一个初始化op将变量设置为初始值。  这事实上是一个tf.assign操作。
  \item 初始值的操作, 例如示例中对biases变量的zeros操作也被加入了graph。
\end{itemize}
\lstinline{tf.Variable}的返回值是Python的\lstinline{tf.Variable}类的一个实例。

\subsection {变量初始化}

变量的初始化必须在模型的其它操作运行之前先明确地完成。  最简单的方法就是添加一个给所有变量初始化的操作, 并在使用模型之前首先运行那个操作。

你或者可以从检查点文件中重新获取变量值, 详见下文。

使用\lstinline{tf.initialize_all_variables()}添加一个操作对变量做初始化。  记得在完全构建好模型并加载之后再运行那个操作。

\begin{lstlisting}
# Create two variables.
weights = tf.Variable(tf.random_normal([784, 200], stddev=0.35),
                      name="weights")
biases = tf.Variable(tf.zeros([200]), name="biases")
...
# Add an op to initialize the variables.
init_op = tf.initialize_all_variables()

# Later, when launching the model
with tf.Session() as sess:
  # Run the init operation.
  sess.run(init_op)
  ...
  # Use the model
  ...
\end{lstlisting}

\subsubsection{由另一个变量初始化}
你有时候会需要用另一个变量的初始化值给当前变量初始化。  由于\lstinline{tf.initialize_all_variables()}是并行地初始化所有变量, 所以在有这种需求的情况下需要小心。

用其它变量的值初始化一个新的变量时, 使用其它变量的\lstinline{initialized_value()}属性。  你可以直接把已初始化的值作为新变量的初始值, 或者把它当做tensor计算得到一个值赋予新变量。

\begin{lstlisting}
# Create a variable with a random value.
weights = tf.Variable(tf.random_normal([784, 200], stddev=0.35),name="weights")
# Create another variable with the same value as 'weights'.
w2 = tf.Variable(weights.initialized_value(), name="w2")
# Create another variable with twice the value of 'weights'
w_twice = tf.Variable(weights.initialized_value() * 0.2, name="w_twice")
\end{lstlisting}

\subsubsection{自定义初始化}

\lstinline{tf.initialize_all_variables()}函数便捷地添加一个op来初始化模型的所有变量。  你也可以给它传入一组变量进行初始化。  详情请见Variables Documentation, 包括检查变量是否被初始化。

% add link of Variables Documentaion here
\subsection {保存和加载}
最简单的保存和恢复模型的方法是使用\lseinline{tf.train.Saver}对象。  构造器给graph的所有变量, 或是定义在列表里的变量, 添加\lstinline{save}和\lstinline{restoreops}。  \lstinline{saver}对象提供了方法来运行这些ops, 定义检查点文件的读写路径。

\subsubsection{Checkpoint Files}
Variables are saved in binary files that, roughly, contain a map from variable names to tensor values.

When you create a Saver object, you can optionally choose names for the variables in the checkpoint files. By default, it uses the value of the \lstinline{Variable.name} property for each variable.
% add link of Variable.name here
% https://www.tensorflow.org/versions/master/api_docs/python/state_ops.html#Variable.name

\subsubsection{保存变量}

用\lstinline{tf.train.Saver()}创建一个\lstinline{Saver}来管理模型中的所有变量。

\begin{lstlisting}
# Create some variables.
v1 = tf.Variable(..., name="v1")
v2 = tf.Variable(..., name="v2")
...
# Add an op to initialize the variables.
init_op = tf.initialize_all_variables()

# Add ops to save and restore all the variables.
saver = tf.train.Saver()

# Later, launch the model, initialize the variables, do some work, save the
# variables to disk.
with tf.Session() as sess:
  sess.run(init_op)
  # Do some work with the model.
  ..
  # Save the variables to disk.
  save_path = saver.save(sess, "/tmp/model.ckpt")
  print "Model saved in file: ", save_path
\end{lstlisting}

\subsubsection{恢复变量}

用同一个\lstinline{Saver}对象来恢复变量。  注意, 当你从文件中恢复变量时, 不需要事先对它们做初始化。

\begin{lstlisting}
# Create some variables.
v1 = tf.Variable(..., name="v1")
v2 = tf.Variable(..., name="v2")
...
# Add ops to save and restore all the variables.
saver = tf.train.Saver()

# Later, launch the model, use the saver to restore variables from disk, and
# do some work with the model.
with tf.Session() as sess:
  # Restore variables from disk.
  saver.restore(sess, "/tmp/model.ckpt")
  print "Model restored."
  # Do some work with the model
  ...
\end{lstlisting}

\subsubsection{选择存储和恢复哪些变量}
如果你不给\lstinline{tf.train.Saver()}传入任何参数, 那么\lstinline{saver}将处理\lstinline{graph}中的所有变量。  其中每一个变量都以变量创建时传入的名称被保存。

有时候在检查点文件中明确定义变量的名称很有用。  举个例子, 你也许已经训练得到了一个模型, 其中有个变量命名为\lstinline{"weights"}, 你想把它的值恢复到一个新的变量\lstinline{"params"}中。

有时候仅保存和恢复模型的一部分变量很有用。  再举个例子, 你也许训练得到了一个5层神经网络, 现在想训练一个6层的新模型, 可以将之前5层模型的参数导入到新模型的前5层中。

你可以通过给\lstinline{tf.train.Saver()}构造函数传入Python字典, 很容易地定义需要保持的变量及对应名称:键对应使用的名称, 值对应被管理的变量。

\textbf{注意}:

\begin{quote}
You can create as many saver objects as you want if you need to save and restore different subsets of the model variables. The same variable can be listed in multiple saver objects, its value is only changed when the saver restore() method is run.

If you only restore a subset of the model variables at the start of a session, you have to run an initialize op for the other variables. See \lstinline{tf.initialize_variables()} for more information.
\end{quote}

如果需要保存和恢复模型变量的不同子集, 可以创建任意多个saver对象。  同一个变量可被列入多个saver对象中, 只有当saver的\lstinline{restore()}函数被运行时, 它的值才会发生改变。

如果你仅在session开始时恢复模型变量的一个子集, 你需要对剩下的变量执行初始化op。  详情请见\lstinline{tf.initialize_variables()}。

% add link of tf.initialize_variables() here
% https://www.tensorflow.org/versions/master/api_docs/python/state_ops.html#initialize_variables

\begin{lstlisting}
# Create some variables.
v1 = tf.Variable(..., name="v1")
v2 = tf.Variable(..., name="v2")
...
# Add ops to save and restore only 'v2' using the name "my_v2"
saver = tf.train.Saver({"my_v2": v2})
# Use the saver object normally after that.
...
\end{lstlisting}
\include{how_tos/c3s02_variable_scope}
\include{how_tos/c3s03_viz_learning}
\include{how_tos/c3s04_graph_viz}
\include{how_tos/c3s05_reading_data}
\include{how_tos/c3s06_threading_and_queues}
\include{how_tos/c3s07_adding_an_op}
\include{how_tos/c3s08_new_data_formats}
\include{how_tos/c3s09_using_gpu}

🍁 % English parts
\newpage
% Chapter 4 API (Python)
\chapter{Python API}
\include{api/c4s00}
\include{api/python/c4s01_framework}
\include{api/python/c4s02_constant_op}
\include{api/python/c4s03_state_ops}
\include{api/python/c4s04_array_ops}
\include{api/python/c4s05_math_ops}
\include{api/python/c4s06_control_flow_ops}
\include{api/python/c4s07_image}
\include{api/python/c4s08_sparse_ops}
\include{api/python/c4s09_io_ops}
\include{api/python/c4s10_python_io}
\include{api/python/c4s11_nn}
\include{api/python/c4s12_client}
\include{api/python/c4s13_train}

\newpage
% Chapter 5 API (C++)
\chapter{C++ API}

\newpage
\chapter{资源}

\newpage
\chapter{其他}

\printindex
\addcontentsline{toc}{chapter}{索引}

\end{document}

